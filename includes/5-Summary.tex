\chapter{总结与展望}
\label{chap:summary}
    虽然我们的数值结果主要针对二维烧绿石薄膜,但我们认为其背后的物理机制和预测现象可推广至其他具有非共线伊辛轴和/或非零Dzyaloshinskii-Moriya相互作用(DMI)的阻挫磁体体系,例如蜂窝晶格~\cite{ganesh2011quantum,joshi2019Mathbb,liu2020featureless}和Kagome晶格~\cite{ma2024upper}。此外,本理论可拓展至三维体材料体系,其中无能隙拓扑态与新奇物理态可能自然涌现~\cite{li2018competing,li2016weyl,hwang2020topological}。另一个值得探索的方向是天然层状Kagome体系,其与烧绿石晶格共享由阻挫导致的诸多特性。诸如赫伯特史密斯石(\ce{ZnCu3(OH)6Cl2})和卡佩拉斯石(\ce{Cu3Zn(OH)6Cl2})等材料作为潜在量子自旋液体候选体系已受到广泛关注~\cite{norman2016ColloquiumHerbertsmithite}。尽管这些体系与我们的二维烧绿石模型存在差异,但它们可能表现出类似的"味"能斯特效应,为理论预测提供理想的实验验证平台。

    为简明地揭示"味"能斯特效应的核心物理机制,本研究基于简化的$S=1$自旋哈密顿量,仅考虑单重态基态与双重简并的第一激发态。尽管真实材料中具有非零自旋自由度的晶体电场(CEF)激发可能更为复杂,但通过第一性原理计算~\cite{brooks1997density}、中子散射~\cite{frick1986crystal,zhang2014neutron}及拉曼光谱~\cite{schaack2007raman}等技术可系统研究CEF能级结构。我们预期本研究的理论框架可应用于候选材料体系而无显著技术障碍。在此背景下,配位环境对CEF激发的形成起关键作用,而应变调控可重构CEF能级~\cite{jensen1991rare,ishikawa2017reversed,pinho2021impact}并间接调制"味"能斯特效应。除外部静态畸变外,晶格本征动力学畸变(即声子)亦可与CEF能级耦合。研究表明:自旋-晶格耦合可在有序磁体中诱导非平庸能带拓扑~\cite{go2019topological,park2019topological,ma2022antiferromagnetic,ma2024Chiral},而声子通过与CEF激发杂化可获得非零角动量~\cite{lujan2024spin,chaudhary2024Giant}。类似机制可能在量子磁态中引发新奇热自旋流响应,此方向值得未来深入研究。
    
    总之,我们理论预言了具有顺磁基态的烧绿石薄膜体系(特别聚焦于二维模型)中存在"味"能斯特效应。该效应源于自旋轨道耦合、晶体电场激发与自旋输运的协同作用,为研究拓扑自旋输运现象提供了独特平台。计算结果表明:"味"能斯特系数呈现温度、各向异性、DMI及磁场依赖的非零响应。CEF激发的贝里曲率对效应强度与符号起决定性作用,暗示其在热自旋流操控与量子顺磁体拓扑输运研究中的潜在应用。我们建议在烧绿石与Kagome磁体等材料中开展实验研究,以验证理论预言并探索量子顺磁体中"味"能斯特效应的新颖物理内涵。