% \iffalse meta-comment
%
% Copyright (C) 2018--2023 by Xiangdong Zeng <xdzeng96@gmail.com>
%
% This work may be distributed and/or modified under the
% conditions of the LaTeX Project Public License, either
% version 1.3c of this license or (at your option) any later
% version. The latest version of this license is in:
%
%   http://www.latex-project.org/lppl.txt
%
% and version 1.3 or later is part of all distributions of
% LaTeX version 2005/12/01 or later.
%
% This work has the LPPL maintenance status `maintained'.
%
% The Current Maintainer of this work is Xiangdong Zeng.
%
% \fi

%*********************************************************************
% fduthesis: 复旦大学论文模板
% 2023-05-27 v0.9a
%
% 重要提示:
%   1. 请确保使用 UTF-8 编码保存
%   2. 请使用 XeLaTeX 或 LuaLaTeX 编译
%   3. 请仔细阅读用户文档
%   4. 修改、使用、发布本文档请务必遵循 LaTeX Project Public License
%   5. 不需要的注释可以尽情删除
%*********************************************************************

\documentclass[type=master]{fduthesis}
% 模板选项:
%   type = doctor|master|bachelor  论文类型,默认为本科论文
%   oneside|twoside                论文的单双面模式,默认为 twoside
%   draft = true|false             是否开启草稿模式,默认关闭
% 带选项的用法示例:
%   \documentclass[oneside]{fduthesis}
%   \documentclass[twoside, draft=true]{fduthesis}
%   \documentclass[type=bachelor, twoside, draft=true]{fduthesis}

\fdusetup{
  % 参数设置
  % 允许采用两种方式设置选项:
  %   1. style/... = ...
  %   2. style = { ... = ... }
  % 注意事项:
  %   1. 不要出现空行
  %   2. “=” 两侧的空格会被忽略
  %   3. “/” 两侧的空格不会被忽略
  %   4. 请使用英文逗号 “,” 分隔选项
  %
  % style 类用于设置论文格式
  style = {
    % font = times,
    % 西文字体(包括数学字体)
    % 允许选项:
    %   font = garamond|libertinus|lm|palatino|times|times*|none
    %
    % cjk-font = none,
    % cjk-font = fandol,
    % 中文字体
    % 允许选项:
    %   cjk-font = adobe|fandol|founder|mac|sinotype|sourcehan|windows|none
    %
    % 注意:
    %   1. 中文字体设置高度依赖于系统。各系统建议方案:
    %        windows:cjk-font = windows
    %        mac:    cjk-font = mac
    %        linux:  cjk-font = fandol(默认值)
    %   2. 除 fandol 和 sourcehan 外,其余字体均为商用字体,请注意版权问题
    %   3. 但 fandol 字体缺字比较严重,而 sourcehan 没有配备楷体和仿宋体
    %   4. 这里中西文字体设置均注释掉了,即使用默认设置:
    %        font     = times
    %        cjk-font = fandol
    %   5. 使用 font = none / cjk-font = none 关闭默认字体设置,需手动进行配置
    %
    % font-size = -4,
    % 字号
    % 允许选项:
    %   font-size = -4|5
    %
    % fullwidth-stop = catcode,
    % 是否把全角实心句点 “.” 作为默认的句号形状
    % 允许选项:
    %   fullwidth-stop = catcode|mapping|false
    % 说明:
    %   catcode   显式的 “。” 会被替换为 “.”(e.g. 不包括用宏定义保存的 “。”)
    %   mapping   所有的 “。” 会被替换为 “.”(使用 LuaLaTeX 编译则无效)
    %   false     不进行替换
    %
    footnote-style = xits,
    % 脚注编号样式
    % 允许选项:
    %   footnote-style = plain|libertinus|libertinus*|libertinus-sans|
    %                    pifont|pifont*|pifont-sans|pifont-sans*|
    %                    xits|xits-sans|xits-sans*
    % 默认与西文字体保持一致
    %
    % hyperlink = color,
    % 超链接样式
    % 允许选项:
    %   hyperlink = border|color|none
    %
    % hyperlink-color = default,
    % 超链接颜色
    % 允许选项:
    %   hyperlink-color = default|classic|material|graylevedl|prl
    %
    bib-backend = bibtex,
    % 参考文献支持方式
    % 允许选项:
    %   bib-backend = bibtex|biblatex
    %
    % bib-style = numerical,
    % 参考文献样式
    % 允许选项:
    %   bib-style = author-year|numerical|<其他样式>
    % 说明:
    %   author-year  著者—出版年制
    %   numerical    顺序编码制
    %   <其他样式>   使用其他 .bst(bibtex)或 .bbx(biblatex)格式文件
    %
    % cite-style = {},
    % 引用样式
    % 默认为空,即与参考文献样式保持一致
    % 仅适用于 biblatex;如要填写,需保证相应的 .cbx 格式文件能被调用
    %
    bib-resource = {main.bib},
    % 参考文献数据源
    % 可以是单个文件,也可以是用英文逗号 “,” 隔开的一组文件
    % 如果使用 biblatex,则必须明确给出 .bib 后缀名
    %
    % logo = {fudan-name.pdf},
    % 封面中的校名图片
    % 模版已自带,通常不需要额外配置
    %
    % logo-size = {0.5\textwidth},      % 只设置宽度
    % logo-size = {{}, 3cm},            % 只设置高度
    % logo-size = {8cm, 3cm},           % 设置宽度和高度
    % 设置校名图片的大小
    % 通常不需要调整
    %
    % declaration-page = {declaration.pdf},
    % 插入扫描版的声明页 PDF 文档
    % 默认使用预定义的声明页,但不带签名
    %
    % auto-make-cover = true
    % 是否自动生成论文封面(封一)、指导小组成员名单(封二)和声明页(封三)
    % 除非特殊需要(e.g. 不要封面),否则不建议设为 false
  },
  %
  % info 类用于录入论文信息
  info = {
    title = {量子顺磁体的自旋能斯特效应},
    % 中文标题
    % 长标题建议使用 “\\” 命令手动换行(不是指在源文件里输入回车符,当然
    % 源文件里适当的换行可以有助于代码清晰):
    %   title = {最高人民法院、最高人民检察院关于适用\\
    %            犯罪嫌疑人、被告人逃匿、死亡案件违法所得\\
    %            没收程序若干问题的规定},
    %
    title* = {Spin Nernst effects in quantum paramagnets},
    % 英文标题
    %
    author = {卢博文},
    % 作者姓名
    %
    % author* = {Your name},
    % 作者姓名(英文 / 拼音)
    % 目前不需要填写
    %
    supervisor = {虞跃 \  教授 \& 陈钢 \  教授},
    % 导师
    % 姓名与职称之间可以用 \quad 打印一个空格
    %
    major = {理论物理},
    % 专业
    %
    degree = academic,
    % 学位类型
    % 允许选项:
    %   degree = academic|professional
    % 说明:
    %   academic      学术学位
    %   professional  专业学位
    %
    department = {物理学系},
    % 院系
    %
    student-id = {21110190043},
    % 作者学号
    %
    % date = {2023 年 1 月 1 日},
    % 日期
    % 注释掉表示使用编译日期
    %
    % secret-level = ii,
    % 密级
    % 允许选项:
    %   secret-level = none|i|ii|iii
    % 说明:
    %   none  不显示密级与保密年限
    %   i     秘密
    %   ii    机密
    %   iii   绝密
    %
    % secret-year = {五年},
    % 保密年限
    % secret-level = none 时该选项无效
    %
    % instructors = {
    %   {虞\quad 跃 \quad 教\quad 授},
    %   {陈\quad 钢 \quad 教\quad 授}
    % },
    % 指导小组成员
    % 使用英文逗号 “,” 分隔
    % 如有需要,可以用 \quad 手工对齐
    %
    keywords = {量子顺磁体, 味能斯特效应, 自旋输运,贝里曲率,拓扑自旋输运。},
    % 中文关键词
    % 使用英文逗号 “,” 分隔
    %
    keywords* = {Quantum Paramagnets, Flavor Nernst Effect, Spin Transport, Berry Curvature, Topological Spin Transport.},
    % 英文关键词
    % 使用英文逗号 “,” 分隔
    %
    clc = {O469},
    % 中图分类号
    %
    % jel = {C02},
    % JEL 分类号,仅适用于经济学院等部分院系
  }
}

% 需要的宏包可以自行调用
\usepackage{physics}
\usepackage[colorlinks=true,citecolor=blue,linkcolor=blue,urlcolor=blue]{hyperref}
%\usepackage{bm}
\usepackage{times}
\usepackage{cases}
\usepackage{wasysym}
\usepackage[version=4]{mhchem}
\usepackage{graphicx}
\usepackage{subfigure}
\usepackage{booktabs}  %provide toprule
\usepackage{multirow}
\usepackage{tikz}
\usepackage[normalem]{ulem}
\usepackage{comment}
\usepackage{lineno}
\usepackage{fduthesis-blind}

% \setCJKmainfont{FZXSSK.TTF}


% 需要的命令可以自行定义
\newcommand{\hilbertH}{\symcal{H}}
\newcommand{\ee}{\symrm{e}}
\newcommand{\ii}{\symrm{i}}
\newcommand{\bm}[1]{\symbf{#1}}

\graphicspath{{figures/}}

\begin{document}

% 这个命令用来关闭版心底部强制对齐,可以减少不必要的 underfull \vbox 提示,但会影响排版效果
% \raggedbottom

% 前置部分包含目录、中英文摘要以及符号表等
\frontmatter

% 目录
\tableofcontents
% 插图目录
\listoffigures
% 表格目录
% \listoftables

\begin{abstract}
  自旋输运这一新兴领域揭示了量子顺磁体引人注目的潜力,尽管它们缺乏长程磁有序,但却具备展现新奇现象的能力,并为先进技术应用开辟了道路。本论文深入探讨了量子顺磁体框架下的能斯特型热自旋输运,重点研究了自旋-轨道耦合与系统基态之间的相互作用,尽管该系统基态是顺磁性的,但可以通过上支激发传递自旋信息。

  利用晶体电场(CEF)激发,我们将上支部分描述为具有不同味的玻色子准粒子,在线性味波理论框架下进行处理。这种方法使我们能够引入味能斯特效应(FNE)的概念,有别于在有序系统中观察到的磁声子能斯特效应,这是一种拓扑自旋输运现象。

  利用包含Dzyaloshinskii-Moriya(DM)相互作用和显著硬轴各向异性的有效 spin-1 哈密顿量,我们通过线性响应理论计算味能斯特系数。我们以研究具有全进全出的伊辛轴配置的二维pyrochlore薄膜为例,进行具体分析。我们阐明了该薄膜中的 FNE 及其对温度、各向异性、DM 相互作用和外部磁场的敏感性。

  论文的结果强调了FNE与CEF激发的贝里曲率之间的内在联系,这一关系为热自旋电流的调控和量子顺磁体中拓扑自旋输运的探索提供了可能的应用前景。我们的结果不仅有助于对量子顺磁体理论的深入理解,而且为未来实验研究奠定了基础,为这些效应在自旋电子学领域的实际应用打下了提供了理论依据。
\end{abstract}

\begin{abstract*}
    The burgeoning field of spin transport research has unveiled the intriguing potential of quantum paramagnets, which lack long-range magnetic order yet harbor the capacity to exhibit novel phenomena and pave the way for advanced technological applications. This thesis delves into the investigation of Nernst-type thermal spin transport within the framework of quantum paramagnets, focusing on the interplay between spin-orbit couplings and the system's ground state, which, despite being paramagnetic, can convey spin information through its upper branch excitations.
    
    By employing the dispersive crystal electric field (CEF) excitations, we describe the upper branch parts in terms of bosonic quasi-particles with distinct flavors within the linear flavor-wave regime. This approach allows us to introduce the concept of the flavor Nernst effect (FNE), a topological spin transport phenomenon that is distinct from magnon Nernst effects observed in ordered systems.
    
    Utilizing an effective spin-1 Hamiltonian that incorporates Dzyaloshinskii-Moriya (DM) interactions and significant hard-axis anisotropy, we calculate the flavor Nernst coefficients through linear response theory. Our analysis is further substantiated by examining a 2D pyrochlore thin film with an all-in-all-out Ising axis configuration, where we elucidate the FNE and its sensitivity to temperature, anisotropy, DM interaction, and external magnetic fields.
    
    The findings of this study underscore the intrinsic link between the FNE and the Berry curvature of the CEF excitations, a relationship that holds promise for the manipulation of thermal spin currents and the exploration of topological spin transport in quantum paramagnets. This research not only contributes to the theoretical understanding of quantum paramagnets but also sets the stage for future experimental endeavors aimed at harnessing these effects for practical applications in the realm of spintronics and thermal management.
\end{abstract*}

% 符号表
% 语法与 LaTeX 表格一致:列用 & 区分,行用 \\ 区分
% 如需修改格式,可以使用可选参数:
%   \begin{notation}[ll]
%     $x$ & 坐标 \\
%     $p$ & 动量
%   \end{notation}
% 可选参数与 LaTeX 标准表格的列格式说明语法一致
% 这里的 “ll” 表示两列均为自动宽度,并且左对齐
% \begin{notation}[ll]

% \end{notation}

% 主体部分是论文的核心
\mainmatter

% 建议采用多文件编译的方式
% 比较好的做法是把每一章放进一个单独的 tex 文件里,并在这里用 \include 导入,例如
%   \include{chapter1}
%   \include{chapter2}
%   \include{chapter3}

\chapter{引言}
\section{自旋能斯特效应}

近年来,热激发自旋电子学(spin caloritronics)作为一门新兴交叉学科,将热流与自旋流的相互转换引入自旋电子学,引发了广泛关注,并在自旋电子学的发展中发挥着至关重要的作用~\cite{zutic2004Spintronics, hoffmann2015Opportunities, barker2021Review, elahi2022Review, nakayama2021Aboveroomtemperature, back2019Special, uchida2021Spintronica, uchida2008Observation}。许多热自旋电子学的现象,如\emph{自旋塞贝克效应}(spin Seebeck effect)~\cite{uchida2008Observationa}和自旋依赖的\emph{帕尔贴效应}(Peltier effect),已被实验观察并在理论上得到了阐释~\cite{flipse2012Direct, bakker2010Interplay, maekawa2017spin, bhardwaj2018Spin, adachi2013Theory, uchida2016Thermoelectric, ma2020longitudinal}。在这些研究中,自旋的拓扑输运已成为一个特别引人注目的领域,为我们提供了对基本量子现象的洞察,并在下一代器件中具有潜在应用~\cite{armitage2018Weyl, tokura2019Magnetic}。其中,\emph{自旋能斯特效应}(Spin Nernst Effect,SNE)是自旋霍尔效应在热力学领域的对应效应,因其独特的物理机制而备受瞩目~\cite{cheng2016Spina, meyer2017Observationb}。简单来说,当在具有强自旋轨道耦合的非磁性金属中施加温度梯度(产生热流)时,会在垂直于热流方向产生横向的纯自旋流,并在样品边界形成自旋累积~\cite{cheng2016spin, zyuzin2016Magnon, meyer2017Observationb, shiomi2017Experimental, sheng2017Spin, ma2021Intrinsic, zhang2022Perspective}。这一效应可类比于\emph{自旋霍尔效应}(spin hall effect, SHE):后者是电场驱动下电流产生横向自旋流,而前者则是温度梯度驱动下热流产生横向自旋流。因此,SNE也常被视为纯自旋流热力学效应,其概念与经典的能斯特效应相呼应,只不过能斯特效应涉及电荷流和横向电压,而自旋能斯特效应涉及自旋流和横向自旋蓄积。作为热激发自旋电子学领域的重要成员,SNE的提出完善了“霍尔家族”中热驱动纯自旋流的缺环,对于理解热流与自旋流的相互转换具有重要意义。

自旋能斯特效应的理论与实验发展脉络清晰可循。早在 2008 年前后,理论研究者便预言在强自旋-轨道耦合体系中存在 SNE~\cite{sheng2017Spin}。此后数年,关于 SNE 的物理机制和潜在材料的理论研究持续深化,但实验验证长期未能突破。直到 2017 年,两个独立实验团队几乎同时报告了 SNE 的首次观测,标志着该领域的重要里程碑。一方面,Meyer 等人~\cite{meyer2017Observationb} 在 \ce{Pt}/\ce{YIG} 异质结构中通过精密的热电测量技术,成功分离出由 SNE 引起的自旋累积信号,证实了铂薄膜在温度梯度作用下可产生横向自旋流。另一方面,Sheng 等人~\cite{sheng2017Spin} 在钨/\ce{CoFeB} 金属双层结构中直接探测到热流激发的自旋流,并提出了“自旋能斯特角”这一参数来量化 SNE 的效率。这两项开创性实验均依托重元素顺磁金属(如 Pt、W)的强自旋霍尔效应背景,实验结果表明,自旋能斯特信号的强度与材料本征自旋霍尔导率的能量导数密切相关。具体而言,钨的自旋能斯特角约为其自旋霍尔角的 70\%,且符号相反。该实验清晰地展示了SNE的存在及其与SHE的对应关系,进一步证明了热力学自旋输运效应的可观测性。

2017年这两项开创性的实验工作填补了自旋能斯特效应实验研究的空白,为理论预言提供了有力佐证,也标志着纯自旋流热效应研究的一个里程碑。SNE的成功观测使得纯自旋流的输运图景更加完整:继自旋霍尔效应、自旋塞贝克效应等之后,自旋能斯特效应作为最后一块拼图被实验证实。这一进展不仅深化了人们对自旋流与热流相互转换规律的认识,也为发展新型自旋热电子器件提供了契机。在随后的研究中,理论工作者进一步完善了描述 SNE 的框架以消除早期的歧义,并将注意力转向探索具有特殊能带结构或更高效率的材料体系。总而言之,自旋能斯特效应从理论预言走向实验实现的历史进程,不仅体现了基础物理概念与应用探索的紧密联系,也凸显了自旋热电效应在自旋电子学与能源科学交叉领域的重要研究意义。

% SNE 在顺磁体系中的研究具有重要的科学意义与应用前景。一方面,相较于传统依赖磁性材料的自旋热效应,SNE 利用无磁有序材料即可生成并输运自旋角动量,为自旋电子学器件的设计提供了新的自由度。例如,在无需外加磁场的条件下,仅通过温度调控即可在重金属薄膜中产生可控的纯自旋流信号,这在自旋电流的能量收集与操控领域具有潜在价值。另一方面,顺磁体系因缺乏长程磁有序,能够避免磁畴噪声和磁阻饱和等干扰因素,从而为研究自旋-轨道耦合的本征性质对热-自旋转换的影响提供了理想平台。然而,SNE 的实验观测面临诸多挑战。首先,热诱导的自旋累积信号通常极其微弱,易被传统塞贝克效应或努斯特效应产生的电信号所掩盖。为此,实验上需设计巧妙的测量方案以分离 SNE 贡献,例如利用磁性绝缘体(如 YIG)的磁化方向依赖性调制自旋流跨界面传输,提取纯净的 SNE 信号。其次,SNE 效应的量化表征(如自旋能斯特电导和自旋能斯特角)需结合自旋霍尔磁阻和逆自旋霍尔效应等多种技术综合分析。这些挑战凸显了顺磁体系中 SNE 研究的复杂性,同时也推动了新型测量方法的发展,例如“自旋能斯特磁阻”技术的提出,为探测反铁磁材料磁化动力学开辟了新路径。

\section{量子磁性概述}\label{sec:quantum_magnetism}
\subsection{磁有序态与顺磁态的比较}

磁性材料的宏观行为通常可分为两类:具有磁有序(magnetically ordered)基态的系统和顺磁(paramagnetic)基态的系统。在磁有序态中,体系在零温下呈现非零的局域磁矩排列,自旋对称性自发破缺。磁有序态的典型类型包括:

\begin{itemize}
    \item \emph{铁磁(Ferromagnetic)}:自旋趋于平行排列,形成净磁化矢量 $\langle \mathbf{S}_i \rangle = \mathbf{M} \neq 0$。
    \item \emph{反铁磁(Antiferromagnetic)}:相邻自旋反平行排列,局域磁矩非零但宏观磁化总和为零。
    \item \emph{复杂磁结构}:如螺旋、自旋密度波或 120° 排列,常见于几何阻挫体系。
\end{itemize}

这些有序态的共同特征是存在非零自旋序参量 $\langle \mathbf{S}_i \rangle \neq 0$,并在低温下通过自发对称性破缺形成,激发通常为 Goldstone 模(如磁振子),可在输运过程中携带热流与自旋流。

与之相对,顺磁态不具备自发有序,自旋在统计平均下无固定取向,满足 $\langle \mathbf{S}_i \rangle = 0$。顺磁态可进一步分为:

\begin{itemize}
    \item \emph{经典顺磁态}:在高温下,热涨落主导,掩盖交换作用,磁矩彼此独立,符合 Curie 定律。
    \item \emph{量子顺磁态}:在低温下,自旋因量子涨落或各向异性被“冻结”于无序基态,但仍存在局域激发。例如,在强单离子各向异性下,$S=1$ 体系可呈现 $\langle S^z_i \rangle = 0$ 的无序基态,而 $\pm 1$ 态作为激发形成色散。
\end{itemize}

磁性源于电子自旋与轨道运动的相互作用。在经典磁性中,如铁磁体和反铁磁体,自旋被视为宏观矢量,在低于居里温度($T_C$)或奈尔温度($T_N$)时自发排列形成磁有序相,可通过平均场近似或 Landau 理论等经典框架描述,特征为存在非零序参量 $\langle \mathbf{S} \rangle \neq 0$。

然而,在低维、低自旋、强量子涨落或阻挫晶格结构下,自旋间的相互作用难以维持经典有序相。此类体系的基态无自发对称性破缺(即 $\langle \mathbf{S} \rangle = 0$),却蕴含丰富的短程关联和非平庸激发行为,这种现象被统称为量子磁性(quantum magnetism)\cite{knolle2019Field,rau2016SpinOrbit}。量子磁性中,经典自旋矢量模型不再适用,需借助量子力学中的自旋算符和希尔伯特空间方法进行分析。

磁有序态通常伴随相变温度,其磁激发行为在相变点附近显著变化。而量子顺磁态在 $T=0$ 时可能存在激发能隙,其激发为玻色型准粒子(如 triplon 或 flavor-wave 模式),可在宽广温度范围内被激活,成为研究拓扑自旋输运的理想平台。本文聚焦于无序但可激发的量子顺磁态,其磁激发在温度梯度下可诱导横向自旋流,进而产生自旋能斯特效应。这一机制与传统磁振子驱动的 SNE 存在本质差异,值得深入探讨。

\subsection{量子涨落与几何阻挫}

量子涨落是量子磁性形成的关键机制。在低自旋体系(如 $S=1/2$)中,海森堡不确定性关系 $\Delta S_x \Delta S_y \geq \frac{1}{2} |\langle S_z \rangle|$ 确保即使在零温下自旋仍有本征涨落。此外,量子隧穿效应使自旋可在不同态间切换,削弱能量偏好的有序排列。

几何阻挫(geometrical frustration)则源于晶格结构导致的自旋相互作用冲突。例如,在三角形或 Kagome 晶格中,相邻自旋若倾向反平行排列,则无法同时满足所有相互作用条件,形成“阻挫”状态。这种阻挫抑制常规有序相,放大量子涨落效应。二者协同作用,可稳定无序但高度纠缠的量子磁态,如量子自旋液体和量子顺磁体。

\subsection{量子磁性模型与拓扑输运}

为阐明量子磁性物理,研究者提出了多种理论模型,常见的包括:

\begin{itemize}
    \item \emph{Heisenberg 模型}:
    \[
    H = J \sum_{\langle i,j \rangle} \mathbf{S}_i \cdot \mathbf{S}_j,
    \]
    其中 $J>0$ 表示反铁磁耦合。该模型在一维 $S=1/2$ 情形下具有精确解(Bethe Ansatz),并呈现量子临界行为。
    \item \emph{Haldane 链模型}:该模型针对一维整数自旋反铁磁链,Haldane 猜想指出 $S=1$ 链具激发能隙,而 $S=1/2$ 链为无隙态 \cite{haldane1983Continuum}。
    \item \emph{Kitaev 模型}:定义于蜂窝晶格的各向异性交换模型,其激发具拓扑性质,是研究自旋液体相的理想模型 \cite{kitaev2006Anyons}。
    \item \emph{单离子各向异性模型}:
    \[
    H_D = D \sum_i (S^z_i)^2,
    \]
    当 $D>0$ 时,基态为 $S^z=0$,形成有能隙的量子顺磁体,是本文分析的基础。
    \item \emph{Dzyaloshinskii-Moriya(DM)相互作用}:
    \[
    H_{DM} = \sum_{\langle i,j \rangle} \mathbf{D}_{ij} \cdot (\mathbf{S}_i \times \mathbf{S}_j),
    \]
    该项源于自旋-轨道耦合,打破空间反演对称,引入手征性和 Berry 曲率,是拓扑磁激发输运的关键 \cite{dzyaloshinsky1958thermodynamic}。
\end{itemize}

量子磁性体系中的自旋激发(如磁振子、triplon 等)在非平庸拓扑能带下,可展现类似电子的拓扑输运现象。相关研究表明:

\begin{itemize}
    \item \emph{磁振子霍尔效应}:已在磁有序体中观测到,激发因 Berry 曲率发生横向偏转,产生热霍尔信号。
    \item \emph{自旋能斯特效应}:在无磁有序的顺磁基态中,拓扑自旋流可由具 Berry 曲率的激发驱动,如本文研究的情形。
    \item \emph{量子化热霍尔效应}:在分数量子自旋液体中,拓扑自旋激发可导致热霍尔导数呈现量子化特征 \cite{kasahara2018Majorana}。
\end{itemize}

量子磁性不仅是基础物理的前沿课题,也成为拓扑物态与自旋电子学的交汇点,其与拓扑输运的关联研究正成为凝聚态物理的重要方向。

    
\section{量子顺磁体}
量子顺磁体的特征是在零温度下缺乏长程磁有序,通常出现在强量子波动抑制了传统磁性有序的材料中~\cite{knolle2019Field}。在许多此类系统中,自旋-轨道耦合(SOC)与晶体电场(CEF)效应之间的相互作用导致了一个具有多个能级的复杂局部希尔伯特空间~\cite{rau2016SpinOrbit}。这些CEF激发可以看作是广义的三重态,类似于二聚化磁体中的三重子激发~\cite{akbari2023Topological, mcclarty2017Topological, romhanyi2015Hall}。

由于这些在量子顺磁体中的相关激发是电荷中性的,传统的电学方法通常在此类情况下失效。相反,热驱动输运已成为研究关联量子材料中基本激发性质的有力探针~\cite{zhang2024Thermal, onose2010Observation, katsura2010theory}。例如,超低温下的纵向热导率的线性残余项归因于可移动的费米激发,已被认为是无间隙量子自旋液体的证据~\cite{ni2019Absence, bourgeois-hope2019Thermal, zhu2023Fluctuating}。另一方面,横向热霍尔效应提供了关于激发贝里曲率的有价值信息~\cite{zhang2024Thermal, ma2024upper, boulanger2020Thermal}。最近的实验揭示了在多种量子磁体中显著的热霍尔信号,包括\ce{Tb2Ti2O7}~\cite{li2013Phononglasslike}、\ce{Yb2Ti2O7}~\cite{tokiwa2016Possible}和Cd-卡佩石\ce{CdCu3(OH)6(NO3)2 * H2O}~\cite{akazawa2020Thermal},其中 CEF 激发发挥了重要作用。除了携带能量外,这些激发还可以通过热驱动力传输自旋信息。

如前所述,磁子自旋能斯特效应要求基态具有磁有序,因此系统温度受相变温度 $T_c$ 的限制。然而,由于是热激发携带非零自旋,基态原则上不需要处于有序相。此外,在许多候选拓扑磁体中,$T_c$通常小于80 K(例如,已有实验的两个拓扑磁子候选材料\ce{CrI3}和\ce{Lu2V2O7}分别具有$T_c\approx 45$ K和$70$ K~\cite{huang2017Layerdependent, onose2010Observation}),而量子顺磁体\ce{Tb2Ti2O7}中的热霍尔信号已经观察到高达142 K~\cite{hirschberger2015Thermal}。因此,我们有动机将自旋能斯特效应的概念扩展到这些量子顺磁体,其中CEF激发具有非零角动量~\cite{babkevich2015Neutrona, thalmeier2024Induced}。与磁性系统中的线性自旋波理论~\cite{kittel2018introduction}相比,我们应用线性味波理论~\cite{joshi1999elementary, li19984}来从顺磁基态获得CEF激发,因此我们将量子顺磁体中线性味波的自旋能斯特效应称为味能斯特效应。

\section{本文内容与结构}
在本文中,我们将详细介绍量子顺磁体中的味能斯特效应。在第~\ref{chap:theory}章中,我们提出了一个有效的自旋-1哈密顿量来实现量子顺磁基态,并引入了用于CEF激发的线性味波表示,同时我们在线性响应理论框架下展示了味能斯特特效应的热响应系数可以通过广义自旋贝里曲率来表示,而这种曲率与CEF激发的能带拓扑结构紧密相关。在第~\ref{chap:pyrochlore_model}章中,我们以二维 pyrochlore 薄膜为例,建立了相应的自旋模型。在第~\ref{chap:Spin_Nernst_Effect}章中,评估了味能斯特系数,并研究了系统温度、耦合强度和外部磁场对味能斯特自旋输运的影响。最后,在第~\ref{chap:summary}章中,我们对本文进行了总结并提出了未来研究的可能方向。



\chapter{量子顺磁体与自旋能斯特效应的理论基础}
\label{chap:theory}
\section{量子顺磁体自旋模型}
一般而言,对于具有\emph{自旋-轨道耦合}(Spin-orbit coupling, SOC)的\emph{莫特绝缘体}(Mott insulator)来说,磁性离子上存在一个局域的总角动量 $\bm{J}$。在单离子极限下,每个离子存在 $2|\bm{J}|+1$ 个能级状态。在\emph{晶体场}(crystal electric field, CEF)和自旋-轨道耦合作用下,这些状态会自然地分裂为基态与若干激发的晶体场多重态。为简单阐明量子顺磁基态与晶体场激发态的基本概念,我们假设基态为单重态,而晶体场第一激发态形成一个二重态,其激发能量为 $\eta>0$,并忽略更高阶晶体场激发态的影响。因此,局域磁矩的三个状态可通过一个有效自旋 $S=1$ 来描述,并在各个格点上都具有各向异性项 $\eta(\hat{\bm{z}}_i\cdot\bm{S}_i)^2$,其中较低的单重态对应于 ${\hat{\bm{z}}_i\cdot\bm{S}_i =0}$,而较高的二重态对应于 ${\hat{\bm{z}}_i\cdot\bm{S}_i = \pm 1}$。这里的 $\hat{\bm{z}}_i$ 是由晶格格点 $i$ 处的局域配位环境决定的局域伊辛轴~\cite{dun2016magnetic,dun2020quantum}。

相应地,相邻晶体场态之间的相互作用可表示为此 spin-1 表象下的自旋交换耦合。由于局域配位环境和由配体介导的电子跃迁作用~\cite{pesin2010mott},除了一般的海森堡交换耦合 $J$ 之外,我们考虑了更一般的有自旋-轨道耦合导致的反对称 Dzyaloshinskii-Moriya (DM) 相互作用 $\bm{D}_{ij}$~\cite{moriya1960anisotropic,dzyaloshinsky1958thermodynamic}的情况。因此,结合局域格点上的单离子各向异性 $\eta$ 相互作用,有效自旋哈密顿量可表示为
\begin{equation}
    H=\sum_{\langle ij\rangle}\left(J \bm{S}_i\cdot\bm{S}_j+\bm{D}_{ij}\cdot\bm{S}_i\times\bm{S}_j\right)+\sum_i\left[\eta(\hat{\bm{z}}_i\cdot\bm{S}_i)^2-B(\hat{\bm{z}}\cdot \hat{\bm{z}}_i)(\hat{\bm{z}}_i\cdot\bm{S}_i)\right],
    \label{H}
\end{equation}
其中,$B$ 为沿全局 $z$ 方向 $\hat{\bm{z}}$ 施加磁场引起的塞曼能级劈裂。

尽管随相互作用参数变化的完整量子基态磁相图难以获得,但在本文中,我们关注具有较大各向异性限制 $\eta>0$ 的情况,此时体系的基态应为每个格点 $i$ 处单重态 $\ket{\hat{\bm{z}}_i\cdot\bm{S}_i=0}$ 的直积态,即
\begin{equation}
    \ket{\text{量子顺磁相}}\equiv\prod_{i} \ket{\hat{\bm{z}}_i\cdot\bm{S}_i=0},
\end{equation}
最低的几个激发态则来自于 $\hat{\bm{z}}_i\cdot\bm{S}_i=\pm 1$,其能量约为 $\eta\pm B$。当其它交换相互作用较小,即 $J,\bm{D}_{ij}\ll\eta$ 时,它们将主要影响激发态的色散能带结构。因此,在温度满足 $k_BT\gtrsim \eta\pm B$ 时,晶体场激发态可以通过热激发进行激活,从而使得这些激发态的热输运成为可能。

\section{线性味波理论}
    为了获得低能激发的色散关系,我们采用了所谓的\emph{线性味波理论}(Linear flavor-wave theory)~\cite{joshi1999elementary,li19984}。由于基态是顺磁的且不是有序的,不能在此应用基于自旋有序系统的 Holstein-Primakoff (HP)变换展开的线性自旋波理论~\cite{holstein1940field}。作为替代,我们将 $\hat{\symbf{z}}_i\cdot\symbf{S}_i=0,\pm 1$ 表示的态(在后文中,我们将$-1$表示为$\bar{1}$)视为三种不同的味,并将系统限制在这些状态所张成的希尔伯特空间中,每个格点 $i$ 的基态为$|f\rangle_i\equiv|\hat{\symbf{z}}_i\cdot\symbf{S}_i=f\rangle$。这些基态组成了 SU(3) 的表示,其生成元为 $G_f^{f'}(i)=|f\rangle_i\langle f'|_i$。可以很容易地注意到 $G_f^{f'}(i)=\left[G_f'^{f}(i)\right]^\dagger$,我们进一步要求其满足归一化条件 $1=G_0^{0}(i)+G_1^{1}(i)+G_{\bar{1}}^{\bar{1}}(i)$。
    
    于是,自旋升降算符可以写为~\cite{ma2024upper}
    \begin{align}
        S_i^+&\equiv(\hat{\symbf{x}}_i+i\hat{\symbf{y}}_i)\cdot\symbf{S}_i\nonumber\\
        &=\sum_{ff'}\langle f|S_i^+|f' \rangle|f \rangle\langle f'|=\sqrt{2}\left[G_1^0(i)+G_0^{\bar{1}}(i)\right],\\
        S_i^-&\equiv(\hat{\symbf{x}}_i-i\hat{\symbf{y}}_i)\cdot\symbf{S}_i\nonumber\\
        &=\sum_{ff'}\langle f|S_i^-|f' \rangle|f \rangle\langle f'|=\sqrt{2}\left[G_{\bar{1}}^0(i)+G_0^1(i)\right].
    \end{align}
    类似地,
    \begin{align}
         S_i^z&\equiv\hat{\symbf{z}}_i\cdot\symbf{S}_i=\sum_{ff'}\langle f|S_i^z|f' \rangle|f \rangle\langle f'|=G_1^1(i)-G_{\bar{1}}^{\bar{1}}(i),\\
        (S_i^z)^2&=\sum_{ff'}\langle f|(S_i^z)^2|f' \rangle|f \rangle\langle f'|=G_1^1(i)+G_{\bar{1}}^{\bar{1}}(i).
    \end{align}
    这里,$(\hat{\symbf{x}}_i,\hat{\symbf{y}}_i,\hat{\symbf{z}}_i)$ 为格点 $i$ 的局部坐标系,因为味是由局部伊辛轴$\hat{\symbf{z}}_i$定义。
    
    根据\emph{杨表}(Young tableaux)~\cite{kim2017linear},SU(3) 代数可以通过两个玻色子 $b$ 和 $\bar{b}$ 来重现,它可以表示为:
    \begin{align}
        G_1^1(i)&=b^\dagger_ib_i,\\
        G_{\bar{1}}^{\bar{1}}(i)&=\bar{b}^\dagger_i\bar{b}_i,\\
        G_0^0(i)&=1-b^\dagger_ib_i-\bar{b}^\dagger_i\bar{b}_i,\\
        G_{\bar{1}}^{1}(i)&=\bar{b}^\dagger_ib_i,\\
        G_1^0(i)&=b^\dagger_i\sqrt{1-b^\dagger_ib_i-\bar{b}^\dagger_i\bar{b}_i}\approx b^\dagger_i,\\
        G_{\bar{1}}^0(i)&=\bar{b}^\dagger_i\sqrt{1-b^\dagger_ib_i-\bar{b}^\dagger_i\bar{b}_i}\approx \bar{b}^\dagger_i,
    \end{align}
    于是我们可以立即得到 $\symbf{S}_i$ 的线性味波表示:
    \begin{align}
        \left\{\begin{array}{lll}
         S_i^z&=b^\dagger_i b_i-\bar{b}^\dagger_i \bar{b}_i,\\
         S_i^-&=\sqrt{2}(\bar{b}^\dagger_i+b_i),\\
         S_i^+&=\sqrt{2}(\bar{b}_i+b^\dagger_i),
        \end{array}
        \right.
    \end{align}
    从物理上讲,玻色子算符 $b^\dagger_i$ 和 $\bar{b}^\dagger_i$($b_i$和$\bar{b}_i$)分别创造(湮灭)量子顺磁基态具有“磁味” $\hat{\symbf{z}}_i\cdot\symbf{S}_i=+1$ 和 $\hat{\symbf{z}}_i\cdot\symbf{S}_i=-1$ 的激发。
    
    一般来说,$\hat{\symbf{z}}_i$ 在每个格点 $i$ 上有不同的取向,因此全局自旋的 U(1) 对称性被破缺。这类似于非共线反铁磁有序相,在这种相中,玻色子算符的配对项是存在的,且需要进行粒子-空穴对偶化来产生Bogoliubov-de Gennes(BdG)哈密顿量~\cite{del2004quantum}。因此,在动量空间中,有 $m$ 个子格的系统哈密顿量方程~\eqref{H}可以写成粒子-空穴对称形式:$H=\frac{1}{2}\sum_{\symbf{k}}\symbf{\Psi}_{\symbf{k}}^\dagger H_{\symbf{k}}\symbf{\Psi}_{\symbf{k}}$,其中
    \begin{gather}
        H_{\symbf{k}}=
        \begin{pmatrix}
            A_{\symbf{k}} & B_{\symbf{k}}\\
            B^*_{-\symbf{k}} & A^*_{-\symbf{k}}
        \end{pmatrix},\label{eq:Ham}\\
        \symbf{\Psi}_{\symbf{k}}\!\!=\!\!\left(b_{1{\symbf{k}}},\!\bar{b}_{1{\symbf{k}}},...,b_{m{\symbf{k}}},\!\bar{b}_{m{\symbf{k}}},\!b^\dagger_{1,-{\symbf{k}}},\!\bar{b}^\dagger_{1,-{\symbf{k}}},...,b^\dagger_{m,-{\symbf{k}}},\!\bar{b}^\dagger_{m,-{\symbf{k}}}\right)^{\mathrm{T}}\!.
    \end{gather}
    味波激发的能带色散可以通过 $\symup{\symup{\Sigma}}_z H_{\symbf{k}}$ 的正特征值来确定,其中
    \begin{equation}
        \symup{\Sigma}_z=
        \begin{pmatrix}
            1 & 0\\
            0 & -1
        \end{pmatrix}
        \otimes I_{2m},
    \end{equation}
    $I_{2m}$ 是 $2m\times 2m$ 的单位矩阵。
    
    \section{线性响应理论}
    已有研究表明,具有 DM 相互作用和 / 或非共线伊辛轴的情况下,CEF 激发可能在时间反演对称性破坏时(如外加磁场),而具有非平凡的拓扑~\cite{ma2024upper}。在来自第 $n$ 个拓扑能带的有限 Berry 曲率 $\symbfup{\Omega}_{n{\symbf{k}}}$ 的作用下,激发的波包将会获得一个横向的反常速度,如下表达式的第二项~\cite{xiao2010berry,cheng2016spin}:
    \begin{align}
        \dot{\symbf{r}}_n=\frac{1}{\hbar}\frac{\partial E_{n{\symbf{k}}}}{\partial {\symbf{k}}}-\dot{{\symbf{k}}}\times\symbfup{\Omega}_{n{\symbf{k}}},
    \end{align}
    其中 $\symbf{r}_n$ 和 $E_{n{\symbf{k}}}$ 分别是第$n$个波函数的包络中心和能带色散,自旋部分的信息是包含在波包中的。由于 CEF 激发还携带“味”,类似于电子的量子自旋霍尔效应~\cite{kane2005quantum} 和拓扑磁子的自旋能斯特效应~\cite{cheng2016spin,zyuzin2016Magnon},如果 $\symbfup{\Omega}_{n{\symbf{k}}}\neq 0$,我们也预期会有\emph{味能斯特效应}(flavor Nernst effect, FNE),即可以通过沿$\beta$方向的纵向温度梯度诱发沿$\alpha$方向流动的具有自旋极化$s$的横向自旋电流,公式如下:
    \begin{align}
        j^s_\alpha=\nu^s_{\alpha\beta}\nabla_\beta T.
    \end{align}
    这里,$\nu^s_{\alpha\beta}$ 是能斯特系数。
    
    通过引入海森堡运动方程~\cite{matsumoto2011theoretical,matsumoto2011rotational,li2020intrinsic},自旋角动量算符 $\tilde{\symbf{S}}$ 的时间演化可以写成:
    \begin{align}
        \frac{\partial}{\partial t}\tilde{\symbf{S}}=i[H,\tilde{\symbf{S}}]=-\nabla\cdot\tilde{\symbf{j}}+\tilde{\symbf{T}},
        \label{eq:Heisenberg equation}
    \end{align}
    其中
    \begin{gather}
        \tilde{\symbf{S}}=\sum_{\symbf{k}}\symbf{\Psi}^\dagger_{\symbf{k}}(S^x,S^y,S^z)\symbf{\Psi}_{\symbf{k}},\\
        S^{s}=\text{Diag}(\hat{\symbf{s}}\cdot\hat{\symbf{z}}_1,\dots,\hat{\symbf{s}}\cdot\hat{\symbf{z}}_m,\hat{\symbf{s}}\cdot\hat{\symbf{z}}_1,\dots,\hat{\symbf{s}}\cdot\hat{\symbf{z}}_m)\otimes
        \begin{pmatrix}
            1 & 0\\
            0 & -1
        \end{pmatrix},
    \end{gather}
    自旋电流算符为
    \begin{equation}
        \tilde{\symbf{j}}^s=\frac{1}{4}\sum_{\symbf{k}}\symbf{\Psi}_{\symbf{k}}^\dagger\left[\symbf{v}_{{\symbf{k}}}\symup{\Sigma}_z S^s+S^s\symup{\Sigma}_z\symbf{v}_{\symbf{k}}\right]\symbf{\Psi}_{\symbf{k}}
    \end{equation}
    其中 $\symbf{v}_{{\symbf{k}}}=\nabla_{\symbf{k}} H_{\symbf{k}}$,方程~\ref{eq:Heisenberg equation}最后一项
    \begin{equation}
        \tilde{T}^s=-\frac{i}{2}\sum_{\symbf{k}}\symbf{\Psi}_{\symbf{k}}^\dagger\left[S^s\symup{\Sigma}_z H_{\symbf{k}}-H_{\symbf{k}}\symup{\Sigma}_z S^s\right]\symbf{\Psi}_{\symbf{k}}
    \end{equation}
    的作用类似一个自旋扭矩算符。在 $s$ 方向极化的自旋电流 $\tilde{\symbf{j}}^s$ 只有在类自旋扭矩项 $\tilde{T}^s=0$ 时才能很好地定义,或者等效地说,此时应该满足 $[\symup{\Sigma}_z\symbf{S}^s, H]=0$。如前所述,这种守恒关系在一般的指向 $\hat{\symbf{z}}_i$ 下可能不成立,此时自旋电流的定义似乎存在问题。然而,线性响应理论的研究表明,在具备反演对称性的系统中,自旋流的热响应仍然可以定义,其中 $\tilde{\symbf{T}}$ 的贡献会因为对称性导致的相消而完全消失~\cite{li2020intrinsic}。对应的能斯特系数 $\nu^s_{\alpha\beta}$ 的表达式为:
    \begin{align}
        \nu^s_{\alpha\beta}=\frac{2k_B}{V}\sum_{n=1}^{2m}\sum_{{\symbf{k}}}\symup{\tilde{\Omega}}^s_{\alpha\beta,n{\symbf{k}}}c_1[g(E_{n{\symbf{k}}})],
        \label{eq:nu}
    \end{align}
    其中 $V$ 是系统的体积,$c_1(x)=(1+x)\ln(1+x)-x\ln x$,$g(x)=(e^{x/k_B T}-1)^{-1}$ 是玻色-爱因斯坦分布,$\symup{\tilde{\Omega}}^s_{\alpha\beta,n{\symbf{k}}}$ 是广义自旋Berry曲率~\cite{li2020intrinsic,ma2021Intrinsic},定义为:
    \begin{equation}
        \symup{\tilde{\Omega}}^s_{\alpha\beta,n{\symbf{k}}}=\sum_{n'\neq n}(\symup{\Sigma}_z)_{nn}\frac{2\text{Im}[(j^s_{\alpha{\symbf{k}}})_{nn'}(\symup{\Sigma}_z)_{n'n'}(v_{\beta{\symbf{k}}})_{n'n}]}{\left[(\symup{\Sigma}_z)_{nn}E_{n{\symbf{k}}}-(\symup{\Sigma}_z)_{n'n'}E_{n'{\symbf{k}}}\right]^2},
        \label{GsBC}
    \end{equation}
    其中 $(\dots)_{nn'}$ 表示 $\langle\psi_{n{\symbf{k}}}|(\dots)|\psi_{n'{\symbf{k}}}\rangle$,$|\psi_{n{\symbf{k}}}\rangle$ 是第 $n$ 个能带的本征波函数,使得
    \begin{equation}
        H_{\symbf{k}} \ket{\psi_{n{\symbf{k}}}} = E_{n{\symbf{k}}} \ket{\psi_{n{\symbf{k}}}},
    \end{equation}
    且 Im[ ] 提取复数的虚部。
    
    在伊辛轴沿 $z$ 方向的共线情形下,即 $\hat{\symbf{z}}_i=\hat{\symbf{z}}$,自旋$z$分量守恒。如果不存在简并,则算符$S^z$和哈密顿量$H_{\bm{k}}$共享相同的本征态$\psi_{\bm{k}}$,满足$S^z |\psi_{n{\bm{k}}}\rangle=s_n^z|\psi_{n{\bm{k}}}\rangle$以及$H_{\bm{k}}|\psi_{n{\bm{k}}}\rangle=E_{n{\bm{k}}}|\psi_{n{\bm{k}}}\rangle$。注意到关系式$\sum_n|\psi_{n{\bm{k}}}\rangle\symup{\Sigma}_z\langle\psi_{n{\bm{k}}}|=\symup{\Sigma}_z$,我们得到
    \begin{align}
        S^z\symup{\Sigma}_z H_{\bm{k}}&=S^z\left(\sum_n|\psi_{n{\bm{k}}}\rangle\symup{\Sigma}_z\langle\psi_{n{\bm{k}}}|\right) H_{\bm{k}}\nonumber\\
        &=\sum_n \left(s_n^z|\psi_{n{\bm{k}}}\rangle\symup{\Sigma}_z\langle\psi_{n{\bm{k}}}| E_{n{\bm{k}}}\right)\nonumber\\
        &=\sum_n \left(E_{n{\bm{k}}}|\psi_{n{\bm{k}}}\rangle\symup{\Sigma}_z\langle\psi_{n{\bm{k}}}|s_n^z\right)\\
        &=H_{\bm{k}}\symup{\Sigma}_z S^z,
    \end{align}
    因此,电流算符满足$j_{\alpha {\bm{k}}}^{z}=\frac{1}{2} S^{z} \symup{\Sigma}_z v_{\alpha {\bm{k}}}$,由此可得
    \begin{align}
        \tilde{\Omega}_{\alpha \beta ,n{\bm{k}}}^{z} &=\sum _{n'\neq n} (\Sigma _{z} )_{nn}\frac{2\text{Im} [(j_{\alpha {\bm{k}}}^{z} )_{nn'} (\Sigma _{z} )_{n'n'} (v_{\beta {\bm{k}}} )_{n'n} ]}{[ (\Sigma _{z} )_{nn} E_{n{\bm{k}}} -(\Sigma _{z} )_{n'n'} E_{n'{\bm{k}}}]^{2}}\nonumber\\
         & =\sum _{n'\neq n} (\Sigma _{z} )_{nn}\frac{2\text{Im} [(\frac{1}{2} S^{z} \Sigma_z v_{\alpha {\bm{k}}} )_{nn'} (\Sigma _{z} )_{n'n'} (v_{\beta {\bm{k}}} )_{n'n} ]}{[ (\Sigma _{z} )_{nn} E_{n{\bm{k}}} -(\Sigma _{z} )_{n'n'} E_{n'{\bm{k}}}]^{2}}\nonumber\\
         & =\frac{1}{2}\sum _{n'\neq n} (\Sigma _{z} )_{nn}\frac{2\text{Im} [(S^{z} \Sigma_z v_{\alpha {\bm{k}}} )_{nn'} (\Sigma _{z} )_{n'n'} (v_{\beta {\bm{k}}} )_{n'n} ]}{[ (\Sigma _{z} )_{nn} E_{n{\bm{k}}} -(\Sigma _{z} )_{n'n'} E_{n'{\bm{k}}}]^{2}}\nonumber\\
         & =\frac{1}{2}\sum _{n'\neq n} (\Sigma _{z} )_{nn}\frac{2\text{Im} [(S^{z} \Sigma_z)_{nm}( v_{\alpha {\bm{k}}} )_{mn'} (\Sigma _{z} )_{n'n'} (v_{\beta {\bm{k}}} )_{n'n} ]}{[ (\Sigma _{z} )_{nn} E_{n{\bm{k}}} -(\Sigma _{z} )_{n'n'} E_{n'{\bm{k}}}]^{2}}\nonumber\\
         & =\frac{1}{2} (S^{z} \Sigma_z)_{nn}\sum _{n'\neq n} (\Sigma _{z} )_{nn}\frac{2\text{Im} [( v_{\alpha {\bm{k}}} )_{nn'} (\Sigma _{z} )_{n'n'} (v_{\beta {\bm{k}}} )_{n'n} ]}{[ (\Sigma _{z} )_{nn} E_{n{\bm{k}}} -(\Sigma _{z} )_{n'n'} E_{n'{\bm{k}}}]^{2}}\nonumber\\
         & =\frac{1}{2}s_n^{z} \Omega_{\alpha \beta ,n{\bm{k}}},
    \end{align}
    其中我们使用了$S^z\symup{\Sigma}_z$在本征基$|\psi_{n{\bm{k}}}\rangle$下为对角矩阵,且粒子能带的$(\symup{\Sigma}_z)_{nn}=+1$的条件,$\symup{\Omega}_{\alpha \beta ,n{\bm{k}}}$ 为常规的 Berry 曲率。由此,我们展示了自旋Berry曲率简化为常规的Berry曲率,并且式~\eqref{eq:nu} 将恢复为研究共线反铁磁体中的自旋能斯特效应的形式~\cite{cheng2016spin,zyuzin2016Magnon}。

    \section{本章小结}
    本章围绕量子顺磁体中的自旋能斯特效应,系统性地建立了其理论基础。首先,我们探讨了量子顺磁体的自旋模型。在具有自旋-轨道耦合的莫特绝缘体中,局域磁矩的状态可通过有效自旋 $S=1$ 表征。结合晶体场效应和自旋-轨道耦合,我们将单离子基态描述为单重态 $\ket{\hat{\bm{z}}_i \cdot \bm{S}_i = 0}$,而第一激发态为二重态 $\ket{\hat{\bm{z}}_i \cdot \bm{S}_i = \pm 1}$,其激发能量由各向异性项 $\eta$ 决定。在此基础上,相邻格点间的自旋交换耦合被纳入模型,包括海森堡交换 $J$ 和 DM 相互作用 $\bm{D}_{ij}$,从而构造了完整的有效哈密顿量。当各向异性 $\eta$ 较大时,系统基态表现为量子顺磁相,其激发态的色散受交换相互作用调制,为后续热输运研究奠定了基础。

    接着,我们引入线性味波理论以分析量子顺磁体中的低能激发。由于基态无序,传统线性自旋波理论不适用,我们采用 SU(3) 表示下的线性味波方法,将自旋态分为三种“味”并通过玻色子算符重构自旋升降算符。在非共线伊辛轴的情况下,味波激发的能带色散通过 Bogoliubov-de Gennes 哈密顿量计算获得,揭示了激发的拓扑性质。这种方法不仅描述了激发态的动态行为,还为自旋能斯特效应的横向输运提供了微观机制。

    最后,我们基于线性响应理论研究了自旋能斯特效应的产生机制。味波激发携带的非平凡拓扑特性通过自旋 Berry 曲率 $\tilde{\Omega}^s_{\alpha\beta,n\bm{k}}$ 体现,在温度梯度作用下诱发横向自旋电流,即味能斯特效应。通过推导能斯特系数 $\nu^s_{\alpha\beta}$ 的表达式,我们展示了其与自旋 Berry 曲率和玻色分布的关系。特别地,在伊辛轴共线的情形下,自旋 Berry 曲率简化为常规 Berry 曲率,与共线反铁磁体中的自旋能斯特效应形式一致,验证了理论框架的普适性。

    综上所述,本章从自旋模型构建、味波激发分析到线性响应计算,系统性地阐述了量子顺磁体中自旋能斯特效应的理论框架,为后续理论推导和数值模拟提供了坚实的理论依据。
\chapter{二维pyrochlore薄层上的模型建立与初步分析}
\label{chap:pyrochlore_model}
    \section{Pyrochlore 的晶格结构}
        几何阻挫的\emph{烧绿石型}(pyrochlore)反铁磁材料,如\ce{Gd2Pt2O7}~\cite{welch2022MagneticStructure}、\ce{NaCdCo2F7}~\cite{kancko2023StructuralSpinglass} 和 \ce{LiGaCr4O8}~\cite{he2021NeutronScattering} 等,是实现量子顺磁相的可能平台之一。为简化起见,我们考虑沿 [111] 方向生长的二维烧绿石薄膜体系(如图~\ref{fig:pyrochlore_lattice} 所示),而非三维体材料。这种准二维烧绿石结构在实际材料中可能实现~\cite{liu2024chiralspinliquidlikestatepyrochlore}。一种较为可行的方法是制备三明治型异质结构,即在绝缘衬底间生长超薄烧绿石层。近年来,分子束外延和脉冲激光沉积技术的进步使得高质量烧绿石薄膜的厚度可达几个晶胞~\cite{fujita2016AllinalloutMagnetic}。研究者还开发了一种新的薄膜原位生长方法,即\emph{重复快速高温合成外延法}(repeated rapid high-temperature synthesis epitaxy, RRHSE)~\cite{kim2019InoperandoSpectroscopic,kim2020StrainEngineeringMagnetic},以改善薄膜应力松弛问题。

        \begin{figure}
            \centering
            \includegraphics[width=0.88\textwidth]{Fig1-Lattice.pdf}
            \caption{(a) 有 AIAO 伊辛轴的单层 pyrochlore 晶格结构。蓝色箭头为每个格点上的局域伊辛轴。红色箭头为两个基矢 $\bm{a}_1=a(1,0)$ 和 $\bm{a}_2=a(1/2,\sqrt{3}/2)$. (b) 对元胞内每个位置进行编号。磁场 $\bm{B}$ 沿着 $[111]$ 方向。图片来自\cite{lu2024Spin}。}
            \label{fig:pyrochlore_lattice}
        \end{figure}

        沿 [111] 轴方向,烧绿石薄膜由五个子晶格构成,形成交替的三角形层和kagome层~\cite{hu2012topological,laurell2017topological}。伊辛轴为体态材料中称为\emph{全进全出}(all-in-all-out, AIAO)的构型~\cite{li2018competing},每个子晶格的局域伊辛轴方向 $\hat{\bm{z}}_i$ 如表~\ref{tab: local coordinate system} 所示。由于体系的空间反演对称性仍保持,热自旋流响应方程~\eqref{eq:nu} 是明确定义的,而键的反演对称性则被打破,使得DM相互作用非零。DM矢量的方向可由Moriya规则确定~\cite{moriya1960anisotropic},例如图~\ref{fig:pyrochlore_lattice} 中键12的DM相互作用为
        \begin{align}
            \bm{D}_{12}=\frac{D}{\sqrt{2}}(-1,1,0),
        \end{align}
        其它键上的 $\bm{D}_{ij}$ 则通过晶格对称性确定。
        \begin{table}
            \centering
            \caption{五个子格的局域坐标系。}
            \begin{tabular}{cccccc}
                \hline
                \hline
                $\mu$ & 1 & 2 & 3 & 4 & 5\\
                $\hat{z}_\mu$ &$\frac{1}{\sqrt{3}}[11\bar{1}]$  &$\frac{1}{\sqrt{3}}[\bar{1}\bar{1}\bar{1}]$  &$\frac{1}{\sqrt{3}}[\bar{1}11]$  &$\frac{1}{\sqrt{3}}[1\bar{1}1]$  &$\frac{1}{\sqrt{3}}[\bar{1}\bar{1}\bar{1}]$\\
                \hline
                \hline
            \end{tabular}
            \label{tab: local coordinate system}
        \end{table}

        由于我们只考虑单层的 pyrochlore 晶格结构,所以晶格的周期性体现在 Kagome 平面上,在 $[111]$ 方向上没有周期性。因此,只需要考虑所有子格投影在 Kagome 平面上的相对位置即可,其定义为 $\bm{\delta}_{ij}=\bm{r}_j-\bm{r}_i$ ,如图~\ref{fig:delta}所示。
        \begin{figure}
            \centering
            \includegraphics[width=0.45\textwidth]{Fig9-Position_vectors.pdf}
            \caption{位矢示意图。图片来自\cite{lu2024Spin}。}
            \label{fig:delta}
        \end{figure}
        它们分别为
        \begin{gather}
            \bm{\delta}_{13}=(\frac{\sqrt{3}}{2},0,0),\  \bm{\delta}_{34}=(-\frac{\sqrt{3}}{4},\frac{3}{4},0),\\
            \bm{\delta}_{41}=(-\frac{\sqrt{3}}{4},-\frac{3}{4},0),\ \bm{\delta}_{12}=(\frac{\sqrt{3}}{4}
            \frac{1}{4},\frac{\sqrt{2}}{2}),\\
            \bm{\delta}_{32}=(-\frac{\sqrt{3}}{4},\frac{1}{4},\frac{\sqrt{2}}{2}),\  \bm{\delta}_{42}=(0,-\frac{1}{2},\frac{\sqrt{2}}{2}), \\
            \bm{\delta}_{i5}=-\bm{\delta}_{i2},\\
            \bm{\delta}_{ij}=-\bm{\delta}_{ji},
        \end{gather}
        其中 $\bm{r}_i$ 是 $i$ 格点的位矢。


    \section{模型建立}
       为研究味能斯特效应,我们将方程~\eqref{eq:nu} 应用于一个具体的单层 pyrochlore 晶格模型。在局域坐标系下,自旋哈密顿量~\eqref{H} 可重写为类似于自旋冰的表达形式~\cite{ross2011quantum}:
        \begin{align}
            H=&\sum_{<ij>} [J_{zz} S^z_i S^z_j + J_{\pm}(S^+_i S^-_j+\text{H.c.})+ J_{\pm\pm}(\gamma_{ij}S^+_i S^+_j + \gamma^*_{ij}S^-_i S^-_j)\nonumber\\
            &+ J_{z\pm}(\xi_{ij} S^z_iS^+_j + \xi_{ij}S^+_iS^z_j+\text{H.c.})] + \sum_i (\eta(S^z_i)^2 - B S^z_i),
            \label{eq:Full_Ham}
        \end{align}
        其中耦合常数为
        \begin{align}
            J_{zz}&=\frac{1}{3}(2\sqrt{2}D-J),\quad J_\pm=-\frac{1}{6}(\sqrt{2}D+J),\nonumber\\
            J_{\pm\pm}&=-\frac{1}{3}(\frac{D}{\sqrt{2}}-J),\quad J_{z\pm}=\frac{1}{6}(D+2\sqrt{2}J),
        \end{align}
        且 $\gamma_{ij}=-\xi^*_{ij}$ 为键依赖的相位变量,若把 $\gamma_{ij}$ 视为矩阵 $\gamma$ 在 $ij$ 处的元素,我们可以把它表示为一个 $5\times5$ 的矩阵
        \begin{equation}
            \gamma=
            \begin{pmatrix} 
                0               &1                   &e^{i2\pi/3}         &e^{-i2\pi/3}     &1\\
                1               &0                   &e^{-i2\pi/3}        &e^{i2\pi/3}      &0\\
                e^{i2\pi/3}     &e^{-i2\pi/3}        &0                   &1                &e^{-i2\pi/3}\\
                e^{-i2\pi/3}    &e^{i2\pi/3}         &1                   &0                &e^{i2\pi/3}\\
                1               &0                   &e^{-i2\pi/3}        &e^{i2\pi/3}      &0
            \end{pmatrix}.                        
        \end{equation}
        
        在量子顺磁相中,利用线性味表象,通过对BdG哈密顿量~\eqref{eq:Ham} 对角化可获得味激发模式。通过傅里叶变换,动量空间下的BdG哈密顿量可表示为
        \begin{align}
            H=\frac{1}{2}\sum_{{\bm{k}}}\bm{\Psi}_{{\bm{k}}}^\dagger H_{{\bm{k}}}\bm{\Psi}_{{\bm{k}}}
            =\frac{1}{2}\sum_{{\bm{k}}}\bm{\Psi}_{{\bm{k}}}^\dagger 
            \begin{pmatrix}
                A_{{\bm{k}}}    & B_{{\bm{k}}}\\
                B^*_{-{\bm{k}}} & A^*_{-{\bm{k}}}
            \end{pmatrix}
            \bm{\Psi}_{{\bm{k}}},
        \end{align}
        其中矩阵元分别为
        \begin{align}
            A_{ij}({\bm{k}}) =
            \begin{cases}
            -\bm{B} \cdot \bm{z}_{i} 
            \begin{pmatrix}
            1 & 0\\
            0 & -1
            \end{pmatrix} 
            +\eta I_2  &,i=j,\\
            0 \cdot I_2  &, (i,j) \in \mathcal{S},\\
            m_{ij} C_{ij}({\bm{k}}) &,\text{其他情况},
            \end{cases}
        \end{align}
        以及
        \begin{align}
            B_{ij}({\bm{k}}) =
            \begin{cases}
            0 \cdot I_{2} & ,i=j\quad \text{且} (i,j) \in \mathcal{S},\\
            n_{ij} C_{ij}({\bm{k}}) & ,\text{其他情况},
            \end{cases}
        \end{align}
        我们定义
        \begin{gather}
            m_{ij} =\frac{1}{3}
            \begin{pmatrix}
            -\sqrt{2} D-J & \left( -\sqrt{2} D +2J\right) \gamma _{ij}\\
            \left( -\sqrt{2} D +2J\right) \gamma _{ij}^{*} & -\sqrt{2} D-J
            \end{pmatrix},\\
            n_{ij} =\frac{1}{3}
            \begin{pmatrix}
            \left( -\sqrt{2} D +2J\right) \gamma _{ij} & -\sqrt{2} D-J\\
            -\sqrt{2} D-J & \left( -\sqrt{2} D +2J\right) \gamma _{ij}^{*}
            \end{pmatrix},\\
            C_{ij}({\bm{k}}) =
            \begin{cases}
            2\cos({\bm{k}} \cdot \delta _{ij}) & ,( i,j) \in \mathcal{P},\\
            e^{i{\bm{k}} \cdot \bm{\delta }_{ij}} & ,\text{其他情况},
            \end{cases}\\
            \mathcal{S}=\{(2,5),(5,2)\},\\
            \mathcal{P}=\{( 1,3) ,( 1,4) ,( 3,4),(3,1),(4,1),(4,3)\}.
        \end{gather}

        为对角化 BdG 哈密顿量,需找到矩阵 $\mathcal{Q}$ 使得$\bm{\Psi}_{{\bm{k}}}=\mathcal{Q}_{{\bm{k}}}\bm{\psi}_{{\bm{k}}}$,并满足$\mathcal{Q}_{{\bm{k}}}^\dagger H_{{\bm{k}}} \mathcal{Q}_{{\bm{k}}}=\mathcal{E}_{{\bm{k}}}$,其中$\mathcal{E}_{{\bm{k}}}$为对角矩阵,其矩阵元即为本征能量:
        \begin{align}
            \mathcal{E}_{{\bm{k}}}=
            \begin{pmatrix}
                E _{1{\bm{k}}} & \cdots & 0 &   &   & \\
                \vdots & \ddots & \vdots & & &\\
                0 & \cdots & E _{2m{\bm{k}}} &   &  &\\
                & & & E_{1,-{\bm{k}}}&\cdots  & 0 \\
                & & & \vdots & \ddots  & \vdots\\
                & & & 0 &\cdots & E_{2m,-{\bm{k}}}
            \end{pmatrix}.
        \end{align}

        为保持玻色对易关系不变 $[\bm{\Psi}^{}_{{\bm{k}}},\bm{\Psi}^\dagger_{{\bm{k}}}]=[\bm{\psi}^{}_{{\bm{k}}},\bm{\psi}^\dagger_{{\bm{k}}}]=\Sigma_z$,有
        \begin{align}
            \Sigma_z=[\bm{\Psi}^{}_{{\bm{k}}},\bm{\Psi}^\dagger_{{\bm{k}}}]=\mathcal{Q}^{}_{{\bm{k}}}[\bm{\psi}^{}_{{\bm{k}}},\bm{\psi}^\dagger_{{\bm{k}}}] \mathcal{Q}^\dagger_{{\bm{k}}} = \mathcal{Q}_{{\bm{k}}} \Sigma_z \mathcal{Q}^\dagger_{{\bm{k}}},
            \label{eq:Normalize_Q}
        \end{align}
        因此$\mathcal{Q}$是\emph{仿酉矩阵}(paraunitary matrix),满足$\mathcal{Q}_{{\bm{k}}}^\dagger = \symup{\Sigma}_z \mathcal{Q}_{{\bm{k}}}^{-1}\symup{\Sigma}_z$,由此得到
        \begin{equation}
            \mathcal{Q}^\dagger_{{\bm{k}}} H_{{\bm{k}}} \mathcal{Q}^{}_{{\bm{k}}} = \mathcal{E}_{{\bm{k}}} \quad
            \Rightarrow \quad \mathcal{Q}^{-1}_{{\bm{k}}} \Sigma_z H_{{\bm{k}}}\mathcal{Q}_{{\bm{k}}}= \Sigma_z \mathcal{E}_{{\bm{k}}}.
        \end{equation}
        因此,BdG哈密顿量的本征能量可以通过计算$\symup{\Sigma}_z H_{{\bm{k}}}$的正本征值获得。
        
        
        图~\ref{fig:dis_pyro} 为参数 $\eta/J = 6$,$D/J=0.06$ 和 $B/J=0.6$ 时的线性味波色散关系。各向异性 $\eta>0$ 完全打开了能隙,外磁场则破坏了时间反演对称性带来的能带简并。此外,与三维体材料相比,二维薄膜烧绿石晶格的立方对称性被打破,从而导致10条完全分离的能带。
        \begin{figure}[!h]
            \centering
            \includegraphics[width=0.86\textwidth]{Fig2-Dispersion.pdf}
            \caption{利用线性味波理论得到的单层 pyrochlore 晶格中 CEF 激发的能带色散。此处我们取 $\eta/J = 6$, $D/J=0.06$ 和 $B/J=0.6$. 图片来自\cite{lu2024Spin}。}
            \label{fig:dis_pyro}
        \end{figure}

    \section{对称性分析}
        在计算能斯特系数之前,我们注意到味能斯特系数张量 $\nu^s_{\alpha\beta}$ 在对称性操作 $\mathcal{R}$ 对应的矩阵表示 $R$ 作用下变换为~\cite{suzuki2017cluster,mook2019thermal,li2020intrinsic}
        \begin{align}
        \nu^s_{\alpha\beta} = \det(R) R_{s s'}R_{\alpha\alpha'}R_{\beta\beta'} \nu^{s'}_{\alpha'\beta'}。
        \end{align}
        因此,能斯特系数张量受晶格对称性的严格约束。
        
        类似于kagome晶格磁振子能斯特效应中的情况~\cite{li2020intrinsic},薄膜烧绿石晶格存在两个关键对称性:即对$yz$平面的镜面反射与时间反演联合操作$\mathcal{M}_{yz} \mathcal{T}$,以及绕$z$轴的三重旋转对称性$\mathcal{C}_{3z}$,要求系数张量的形式为
        \begin{align}
        [\nu^x\!,\nu^y\!,\nu^z]=\begin{bmatrix}
            \begin{pmatrix}
            -\nu^y_{xy}     & 0\\
            0                  & \nu^y_{xy}\\
        \end{pmatrix}\!,\begin{pmatrix}
            0                  & \nu^y_{xy}\\
            \nu^{y}_{xy}    & 0
        \end{pmatrix}\!\begin{pmatrix}
            0                  & \nu^z_{xy}\\
            -\nu^z_{xy}     & 0
        \end{pmatrix}
        \end{bmatrix}。
        \end{align}
        因此,在此二维薄膜烧绿石体系中,仅存在两个独立的能斯特系数$\nu^y_{xy}$和$\nu^z_{xy}$。值得注意的是,尽管经典情况下全进全出伊辛轴构型使得$\sum_i\langle S_i^y \rangle=0$,但自旋-$y$方向的横向输运在对称性上仍然是允许的。
        
    \section{本章小结}

    在本章中,我们对二维 pyrochlore 薄层上的模型进行了系统性的建立与初步分析。首先,我们详细描述了 pyrochlore 的晶格结构,聚焦于沿 [111] 方向生长的二维薄膜体系。研究表明,该体系由五个子晶格构成,形成交替的三角形层和 kagome 层,伊辛轴呈现 AIAO 的构型。通过 Moriya 规则,我们进一步确定了 DM 相互作用的方向,为后续模型构建提供了结构基础。

    随后,我们建立了描述该体系的自旋哈密顿量,并将其改写为类似于自旋冰的表达形式,以简化分析过程。通过线性味表象和 BdG 哈密顿量的对角化,我们计算了味激发的能带色散关系。研究结果表明,在特定参数(如 $\eta/J = 6$、$D/J = 0.06$ 和 $B/J = 0.6$)下,各向异性 $\eta$ 和外磁场 $B$ 对能带结构产生显著影响:各向异性完全打开能隙,而外磁场则打破时间反演对称性导致的能带简并,形成了10条分离的能带。

    此外,我们进行了对称性分析,探讨了味能斯特系数张量在晶格对称性约束下的行为。结合镜面反射与时间反演联合操作 $\mathcal{M}_{yz} \mathcal{T}$ 以及三重旋转对称性 $\mathcal{C}_{3z}$,我们得出结论:在二维 pyrochlore 薄膜体系中,仅存在两个独立的能斯特系数 $\nu^y_{xy}$ 和 $\nu^z_{xy}$。这一结果揭示了体系对称性对横向热输运性质的重要影响。

    综上所述,本章通过晶格结构分析、模型构建以及能带和对称性计算,系统研究了二维 pyrochlore 薄层体系的物理性质,为理解其味激发和能斯特效应奠定了坚实的理论基础,同时为后续深入探索提供了必要的前提和方向。
\chapter{二维pyrochlore薄层上的自旋能斯特效应}
\label{chap:Spin_Nernst_Effect}
    \section{味能斯特系数与温度的变化关系}
        在图~\ref{fig:dis_pyro} 采用相同的参数选择下,我们在图~\ref{fig:coe_with_T} 中绘制了两种不同味的能斯特系数 $\nu^y_{xy}$ 和 $\nu^z_{xy}$ 随温度的变化情况。我们发现 $\nu^y_{xy}$ 确实非零,但比 $\nu^z_{xy}$ 低一个数量级,这反映了在非共线伊辛轴配置下,自旋非守恒效应导致 $S^y_{n{\bm{k}}}=\langle\psi_{n{\bm{k}}}|S^y|\psi_{n{\bm{k}}}\rangle$ 尽管很小但仍然是有限的。
        \begin{figure}[h]
            \centering
            \includegraphics[width=0.88\textwidth]{Fig3-TFNC.pdf}
            \caption{(a) 味能斯特系数 $\nu^y_{xy}$ 和 $\nu^z_{xy}$ 随温度的变化。设定 $\eta/J=6$,$D/J=0.06$,$B/J=0.6$。$\nu^y_{xy}$ 比 $\nu^z_{xy}$ 小一个数量级。 (b) $\mathcal{D}^y_{xy}(E) c_1[g(E/k_B T)]$ 随 $k_B T/J$ 和 $E/J$ 的变化。图中垂直虚线对应 $k_B T/J=0.58$,$k_B T/J=1.05$ 和 $k_B T/J=1.75$。图片来自\cite{lu2024Spin}。}
            \label{fig:coe_with_T}
        \end{figure}
    
        整体而言,随着温度升高,味能斯特系数的数值增加,因为方程~\eqref{eq:nu} 中的 $c_1$ 函数在较高温度下取值更大。然而,随温度升高,高能带的自旋Berry曲率贡献变得更加不可忽略,从而导致 $\nu^y_{xy}$ 呈现非单调行为。
        
        为了更详细地理解温度效应,我们在图~\ref{fig:colored dispersion and DOsB} 中通过
        \begin{equation}
            L(\symup{\tilde{\Omega}}^s_{xy})=\text{sgn}(\symup{\tilde{\Omega}}^s_{xy})\ln(1+|\symup{\tilde{\Omega}}^s_{xy}|)
        \end{equation}
        对能带进行着色,其中 $L(x)=\text{sgn}(x)\ln{(1+|x|)}$ 为对数函数。结果显示,自旋Berry曲率的非零值主要集中在反交叉点处,并且两个能带的自旋Berry曲率符号相反。

        \begin{figure}[b]
            \centering
            \includegraphics[width=\textwidth]{Fig4-DOsB.pdf}
            \caption{由对数化的自旋 Berry 曲率 $L(\tilde{\Omega}^s_n)$ 着色的色散关系,以及 DOS $\mathcal{D}(E)$ 和 DOsB $\mathcal{D}^s_{\alpha \beta}(E)$。(a) 为 $s=y$ 的情况,(b) 为 $s=z$ 的情况。图片来自\cite{lu2024Spin}。}
            \label{fig:colored dispersion and DOsB}
        \end{figure}
        
        在图~\ref{fig:colored dispersion and DOsB} 中,我们还展示了\emph{态密度}(Density of states, DOS) $\mathcal{D}(E)$ 以及\emph{自旋Berry曲率密度}(Density of spin Berry curvature, DOsB) $\mathcal{D}^s_{\alpha\beta}(E)$,其定义如下:
        \begin{align}
            \mathcal{D}(E)&= \int_{BZ} \sum_n \frac{d{\bm{k}}}{(2\pi)^2} \delta(E - E_n({\bm{k}})),\\
            \mathcal{D}^s_{\alpha\beta}(E)&= \int_{BZ} \sum_n \frac{d{\bm{k}}}{(2\pi)^2} \delta(E - E_n({\bm{k}})) \symup{\tilde{\Omega}}^s_{\alpha\beta,n{\bm{k}}}.
        \end{align}
        由于低能带在玻色子统计下对 味能斯特效应 贡献更为显著,并且最低的四条能带在能量上与更高能带分离,因此我们主要关注这四条能带。可以清晰地看到在4.8$J$以下DOsB呈现交替符号的显著变化,从而导致 $\nu^y_{xy}$ 的非单调行为。
        
        在数值计算 DOS 和 DOsB 时,由于程序上无法实现 $\delta(x)$ 函数,故我们选择使用另一个函数来近似 $\delta(x)$ 函数。在本文中,我们选取
        \begin{equation}
            \delta(x-x_0) \sim \frac{1}{2\sigma } e^{-\frac{\pi ( x-x_{0})^{2}}{\sigma ^{2}}}
        \end{equation}
        这是一个变形的高斯函数。当 $\sigma$ 较小的时候,上式有着非常好的近似效果。但是需要注意,$\sigma$ 不是越小越好,需要选择适当的值。如果 $\sigma$ 过小,会导致图像不光滑,产生很多锯齿。如果 $\sigma$ 过大,则无法很好地满足 $\delta$ 函数的筛选性质,使得计算不够准确。经过估算,$\delta$ 函数的宽度在 $2dk$ 时是最佳的。在本文中,我们取 $\sigma = 1.5 \times 10^{-2}$.

        从图~\ref{fig:coe_with_T}(a) 可以看出,$\nu^y_{xy}$ 在达到某一温度后开始显著增加,并在 $k_B T/J=0.58$ 处达到最大增长速率。随后,随着温度升高,该系数在 $k_B T/J=1.05$ 处达到峰值,然后下降,在 $k_B T/J=1.75$ 处达到局部最小值,之后又逐渐回升。为了解释 $\nu^y_{xy}$ 的行为,我们在图~\ref{fig:coe_with_T}(b) 中绘制了 $\mathcal{D}^y_{xy}(E) c_1[g(E/k_B T)]$ 随 $k_B T/J$ 和 $E/J$ 的变化情况。对于 $\mathcal{D}^y_{xy}(E)$,在 $3.34 <E/J <3.79$ 范围内主要为正(红色),在 $3.79< E/J <4.41$ 范围内主要为负(蓝色),如图~\ref{fig:colored dispersion and DOsB} 所示。因此,我们利用这些关键数据点在图~\ref{fig:coe_with_T}(b) 中用虚线标记了几个区域。当 $k_B T/J<0.58$ 时,我们在任意能量水平都未观察到蓝色区域。当温度处于 $0.58< k_B T/J <1.05$ 时,负区域IV减缓了 $\nu^y_{xy}$ 的增长。当温度进入 $1.05< k_B T/J <1.75$ 区间时,负区域V(尤其是围绕 $E/J=4$ 的深蓝区域)逆转了 $\nu^y_{xy}$ 的增长趋势。随着温度进一步升高,区域IX和III逐渐变红,导致 $\nu^y_{xy}$ 恢复增长趋势。

        
    \section{味能斯特系数与各项异性相互作用的变化关系}
        除了 $k_B T$ 之外,另一个特征能量尺度是各向异性 $\eta$,它决定了从顺磁基态到晶体场激发态的能隙。如图~\ref{fig:D0.06_B0.6_eta} 所示,较小的 $\eta$ 使得 CEF 能级更容易被热激发,从而导致更强的热响应。以焦绿石材料的典型晶格常数 $d\sim 10$ \AA~\cite{wen2021epitaxial} 作为薄膜厚度的估计,并考虑温度梯度 $\nabla_z T\sim 10$ K/mm~\cite{lin2022Evidence} 及图~\ref{fig:D0.06_B0.6_eta} 所采用的相同参数,可得到实验可观测的 $z(y)$ 极化自旋流,其大小可达 $10^{-11}$ ($10^{-12}$) J/m$^2$。

        \begin{figure}[th]
            \centering
            \includegraphics[width=0.9\textwidth]{Fig5-etaFNC.pdf}
            \caption{在 $k_B T/J=2$ 时,味能斯特系数(单位为 $k_B$)随 $\eta$ 的变化情况。其中 $D/J=0.06$,$B/J=0.6$。图片来自\cite{lu2024Spin}。}
            \label{fig:D0.06_B0.6_eta}
        \end{figure}
        
        显然,除了能量尺度的影响之外,能带的自旋 Berry 曲率本身也会显著影响 味能斯特效应。如前所述,在共线情况下,自旋 Berry 曲率与 Berry 曲率直接相关,因此即使在更一般的非共线情况下,味能斯特效应 也应当反映能带的几何结构。例如,如果重现由动量分辨的自旋 $z$ 分量 $S^z_{n{\bm{k}}} \equiv \langle \psi_{n{\bm{k}}} | S^z | \psi_{n{\bm{k}}} \rangle$ 进行着色的能带色散(如图~\ref{fig:Sz-Dispersion} 所示),则可以观察到,较低的三个能带主要为自旋向上态。因此,自旋 Berry 曲率可近似由能带 Berry 曲率给出:
        \begin{equation}
            \tilde{\Omega}^z_{xy,n{\bm{k}}} \sim \left( \frac{1}{V} \sum_{\bm{k}} S^z_{n{\bm{k}}} \right) \Omega_{xy,n{\bm{k}}},
        \end{equation}
        从而 味能斯特效应 间接地反映了能带的几何与拓扑特性。

        \begin{figure}[bht]
            \centering
            \includegraphics[width=0.9\textwidth]{Fig6-Sz_Dispersion.pdf}
            \caption{由动量分辨的自旋 $z$ 分量 $\bra{\psi_{n{\bm{k}}}} S^z \ket{\psi_{n{\bm{k}}}}$ 进行着色的能带色散关系。图片来自\cite{lu2024Spin}。}
            \label{fig:Sz-Dispersion}
        \end{figure}
        
    \section{味能斯特系数与DM相互作用的变化关系}
        由于Dzyaloshinskii-Moriya(DM)相互作用可以显著改变能带拓扑,我们在图~\ref{fig:dmi_with_t} 中研究了不同DM相互作用强度 $D$ 下 $\nu^{z}_{xy}$ 随温度 $T$ 的变化关系。

        \begin{figure}[th]
            \centering
            \includegraphics[width=0.83\textwidth]{figures/Fig7-DFNC.pdf}
            \caption{不同参数下的能斯特系数。我们设定 $\eta/J=6$,$B/J=0.6$,并考察不同 $D/J$ 的情况。同时,我们标注了最低四条能带的Chern数。图片来自\cite{lu2024Spin}。}
            \label{fig:dmi_with_t}
        \end{figure}
        
        对于每条曲线,从左到右的四个数值表示自底向上最低四条能带的Chern数,并且这些数值与曲线的行为相匹配。例如,当 ${D/J=-0.2}$ 或 $-0.1$ 时,最低能带具有正的Chern数 $+3$,从而导致正的能斯特系数。然而,与 $D/J=-0.2$ 时产生的较大正响应相比,在 $D/J=-0.1$ 的情况下,由于第二条能带的Chern数为 $-3$,较大的负值抑制了 $\nu^z_{xy}$ 随温度升高的增长。
        
        相反,当 $D/J=0$,$0.1$ 和 $0.2$ 时,最低能带的Chern数为 $-1$,从而导致负的能斯特系数。值得注意的是,由于伊辛轴的非共线性,即使在 $D=0$ 时,能带的Chern数以及能斯特系数仍然不为零。由于DM相互作用的微小偏差能够显著改变能带拓扑及热响应,因此在应变工程的帮助下,可以较容易地控制热自旋电流,特别是在层状系统或异质结构中~\cite{kim2020StrainEngineeringMagnetic,zhang2021strain,xu2022strain}。在这些情况下,沿层状方向施加的应变不仅可以调控体相或界面DM矢量,还能调整伊辛轴的方向。因此,我们的理论为操控热自旋电流提供了一条潜在路径,并可能在压电自旋电子学(piezo-spintronics)领域找到应用~\cite{nunez2014theory,ulloa2017piezospintronic,liu2019antiferromagnetic}。
        
    \section{味能斯特系数与外磁场的变化关系}
        由于磁场不仅可以改变能带色散关系,还会影响自旋Berry曲率的分布,因此味能斯特效应也可以通过磁场进行调控。然而,与热霍尔效应不同的是,反转磁场并不会翻转热自旋电流的响应,这是因为自旋Berry曲率在时间反演对称性下保持不变。

        在图~\ref{fig:Dz6_D0.06_B_T} 中,我们展示了外加磁场 $B$(以 $J$ 为单位)下 $
        nu^y_{xy}$ 和 $
        nu^z_{xy}$ 在不同温度下的变化关系。虚线部分是在 $B=0$ 附近的外推数据,以节省在极小能隙下的计算时间。由于在小磁场趋于零的过程中,没有其他能隙闭合或拓扑相变发生,因此这些外推数据应当能够很好地反映该系数变化的趋势。我们的计算结果表明,响应系数 $
        nu^y_{xy}$ 和 $
        nu^z_{xy}$ 确实是 $B$ 的偶函数,并且当 $B\to0$ 时,它们的取值不一定趋于零。
        \begin{figure}[htp]
            \centering
            \includegraphics[width=0.6\textwidth]{figures/Fig8-BFNC.pdf}
            \caption{不同温度 $k_B T/J$ 下的 flavor Nernst 系数随磁场 $B$ 的变化。(a) 对应于 $\nu^y_{xy}$,(b) 对应于 $\nu^z_{xy}$。参数取值为 $\eta/J=6$,$D/J=0.06$。虚线表示计算数据的外推结果。图片来自\cite{lu2024Spin}。}
            \label{fig:Dz6_D0.06_B_T}
        \end{figure}

    \section{本章小结}

    在本章中,我们系统研究了二维 \textit{pyrochlore} 薄层上的自旋能斯特效应,并深入探讨了味能斯特系数随温度、各项异性相互作用、DM 相互作用以及外磁场的变化规律及其物理机制。

    首先,我们在非共线伊辛轴配置下,分析了味能斯特系数 $\nu^y_{xy}$ 和 $\nu^z_{xy}$ 随温度的变化特征。研究表明,$\nu^y_{xy}$ 虽非零,但其数值比 $\nu^z_{xy}$ 小一个数量级,这源于自旋非守恒效应的影响。通过计算自旋 Berry 曲率在能带中的分布,并结合态密度(DOS)和自旋 Berry 曲率密度(DOsB)的分析,我们揭示了 $\nu^y_{xy}$ 非单调行为的来源,即低能带和高能带的自旋 Berry 曲率贡献随温度变化的竞争效应。

    其次,我们考察了味能斯特系数与各项异性相互作用 $\eta$ 的依赖关系。结果显示,较小的 $\eta$ 减小了晶体场激发能隙,使能级更容易被热激发,从而显著增强热响应。同时,自旋 Berry 曲率与能带几何结构的关联表明,味能斯特效应间接反映了系统的拓扑特性。

    接着,我们研究了 Dzyaloshinskii-Moriya(DM)相互作用对味能斯特系数的影响。DM 相互作用通过改变能带拓扑结构显著调控了 $\nu^z_{xy}$ 的行为。我们发现,不同 DM 强度下 $\nu^z_{xy}$ 随温度的变化与最低四条能带的 Chern 数密切相关,为通过应变工程操控热自旋电流提供了理论依据。

    最后,我们分析了外磁场对味能斯特系数的影响。计算结果表明,$\nu^y_{xy}$ 和 $\nu^z_{xy}$ 是磁场 $B$ 的偶函数,且在 $B \to 0$ 时不一定趋于零。这一特性源于自旋 Berry 曲率在时间反演对称性下的不变性,区别于热霍尔效应的行为。

    综上所述,本章通过数值计算和理论分析,揭示了二维 \textit{pyrochlore} 薄层中自旋能斯特效应的多场调控机制,阐明了温度、各项异性、DM 相互作用和外磁场对味能斯特系数的综合影响。研究成果不仅加深了我们对自旋能斯特效应物理本质的理解,还为通过外部参数调控热自旋电流提供了理论支持,具有潜在的压电自旋电子学应用价值。
\chapter{总结与展望}
\label{chap:summary}
    虽然我们的数值结果主要针对二维烧绿石薄膜,但我们认为其背后的物理机制和预测现象可推广至其他具有非共线伊辛轴和/或非零Dzyaloshinskii-Moriya相互作用(DMI)的阻挫磁体体系,例如蜂窝晶格~\cite{ganesh2011quantum,joshi2019Mathbb,liu2020featureless}和Kagome晶格~\cite{ma2024upper}。此外,本理论可拓展至三维体材料体系,其中无能隙拓扑态与新奇物理态可能自然涌现~\cite{li2018competing,li2016weyl,hwang2020topological}。另一个值得探索的方向是天然层状Kagome体系,其与烧绿石晶格共享由阻挫导致的诸多特性。诸如赫伯特史密斯石(\ce{ZnCu3(OH)6Cl2})和卡佩拉斯石(\ce{Cu3Zn(OH)6Cl2})等材料作为潜在量子自旋液体候选体系已受到广泛关注~\cite{norman2016ColloquiumHerbertsmithite}。尽管这些体系与我们的二维烧绿石模型存在差异,但它们可能表现出类似的"味"能斯特效应,为理论预测提供理想的实验验证平台。

    为简明地揭示"味"能斯特效应的核心物理机制,本研究基于简化的$S=1$自旋哈密顿量,仅考虑单重态基态与双重简并的第一激发态。尽管真实材料中具有非零自旋自由度的晶体电场(CEF)激发可能更为复杂,但通过第一性原理计算~\cite{brooks1997density}、中子散射~\cite{frick1986crystal,zhang2014neutron}及拉曼光谱~\cite{schaack2007raman}等技术可系统研究CEF能级结构。我们预期本研究的理论框架可应用于候选材料体系而无显著技术障碍。在此背景下,配位环境对CEF激发的形成起关键作用,而应变调控可重构CEF能级~\cite{jensen1991rare,ishikawa2017reversed,pinho2021impact}并间接调制"味"能斯特效应。除外部静态畸变外,晶格本征动力学畸变(即声子)亦可与CEF能级耦合。研究表明:自旋-晶格耦合可在有序磁体中诱导非平庸能带拓扑~\cite{go2019topological,park2019topological,ma2022antiferromagnetic,ma2024Chiral},而声子通过与CEF激发杂化可获得非零角动量~\cite{lujan2024spin,chaudhary2024Giant}。类似机制可能在量子磁态中引发新奇热自旋流响应,此方向值得未来深入研究。
    
    总之,我们理论预言了具有顺磁基态的烧绿石薄膜体系(特别聚焦于二维模型)中存在"味"能斯特效应。该效应源于自旋轨道耦合、晶体电场激发与自旋输运的协同作用,为研究拓扑自旋输运现象提供了独特平台。计算结果表明:"味"能斯特系数呈现温度、各向异性、DMI及磁场依赖的非零响应。CEF激发的贝里曲率对效应强度与符号起决定性作用,暗示其在热自旋流操控与量子顺磁体拓扑输运研究中的潜在应用。我们建议在烧绿石与Kagome磁体等材料中开展实验研究,以验证理论预言并探索量子顺磁体中"味"能斯特效应的新颖物理内涵。

% 附录部分
\appendix

\chapter{扭矩响应系数}
本节中,我们将说明方程~\eqref{eq:Heisenberg equation}中力矩项的热响应实际上为零,因此即使自旋U(1)旋转对称性被破坏,热自旋流依然是良定义的。

方程~\eqref{eq:Heisenberg equation}中的力矩项为
\begin{equation}
    \tilde{T}^s=-\frac{i}{2}\sum_{\bm{k}}\bm{\Psi}{\bm{k}}^\dagger\left[S^s\Sigma_z H{\bm{k}}-H_{\bm{k}}\Sigma_z S^s\right]\bm{\Psi}{\bm{k}}。
\end{equation}
类似于方程~\eqref{eq:nu},我们可以通过将$\hat{\bm{j}}^s$替换为$\hat{T}^s$,计算由热流引起的力矩响应。根据线性响应理论,力矩响应系数$(\nu_T)^s_y=T^s/\nabla_y T$可表示为
\begin{align}
    (\nu_T)^s{\alpha}=\frac{2k_B}{V}\sum_{n=1}^{2m}\sum_{{\bm{k}}}(\tilde{\Omega}T)^s{\alpha\beta,n{\bm{k}}}c_1[g(E_{n{\bm{k}}})],
    \label{eq:nu_T}
\end{align}
其中$(\tilde{\Omega}T)^s{\alpha,n{\bm{k}}}$定义为
    \begin{equation}
    (\tilde{\Omega}T)^s{\alpha,n{\bm{k}}}=\sum_{n'\neq n}(\Sigma_z){nn}\frac{2\text{Im}[(T^s{{\bm{k}}}){nn'}(\Sigma_z){n'n'}(v_{\beta{\bm{k}}}){n'n}]}{\left[(\Sigma_z){nn}E_{n{\bm{k}}}-(\Sigma_z){n'n'}E{n'{\bm{k}}}\right]^2}。
    \label{GsBC_T}
\end{equation}

图\ref{fig:Torque}(a)和\ref{fig:Torque}(b)表明力矩项的广义Berry曲率$(\tilde{\Omega}T)^s{\alpha,n{\bm{k}}}$确实满足关系$(\tilde{\Omega}T)^s{\alpha,n{\bm{k}}}=-(\tilde{\Omega}T)^s{\alpha,n\bm{-k}}$。图\ref{fig:Torque}(c)和\ref{fig:Torque}(d)则显示该系数为零。

\begin{figure}[!h]
    \centering
    \includegraphics[width=0.88\textwidth]{Fig10-Torque.pdf}
    \caption{设定参数为 $\eta/J=6, D/J=0.06, B/J=0.6$。(a) 为自旋方向 $s=y$ 时力矩项的广义Berry曲率图,展示的是第一能带的情况。(b) 为自旋方向 $s=z$ 时第一能带的广义 Berry 曲率图。(c) 和 (d) 分别展示了系数 $\nu^y_{xy}$和$\nu^z_{xy}$ 随温度的变化,这些系数实际上均为零。图片来自\cite{lu2024Spin}。}
    \label{fig:Torque}
\end{figure}

% 后置部分包含参考文献、声明页(自动生成)等
\backmatter

% 打印参考文献列表
\printbibliography

\chapter{硕士期间发表的论文}
第一作者,Physical Review B(2024)

\begin{acknowledgements}
  感谢各位老师的帮助和支持,感谢各位师兄对我的帮助。
\end{acknowledgements}

\end{document}
