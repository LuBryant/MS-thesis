\chapter{二维pyrochlore薄层上的自旋能斯特效应}
\label{chap:Spin_Nernst_Effect}
    \section{味能斯特系数与温度的变化关系}
        在图~\ref{fig:dis_pyro} 采用相同的参数选择下,我们在图~\ref{fig:coe_with_T} 中绘制了两种不同味的能斯特系数 $\nu^y_{xy}$ 和 $\nu^z_{xy}$ 随温度的变化情况。我们发现 $\nu^y_{xy}$ 确实非零,但比 $\nu^z_{xy}$ 低一个数量级,这反映了在非共线伊辛轴配置下,自旋非守恒效应导致 $S^y_{n{\bm{k}}}=\langle\psi_{n{\bm{k}}}|S^y|\psi_{n{\bm{k}}}\rangle$ 尽管很小但仍然是有限的。
        \begin{figure}[h]
            \centering
            \includegraphics[width=0.88\textwidth]{Fig3-TFNC.pdf}
            \caption{(a) 味能斯特系数 $\nu^y_{xy}$ 和 $\nu^z_{xy}$ 随温度的变化。设定 $\eta/J=6$,$D/J=0.06$,$B/J=0.6$。$\nu^y_{xy}$ 比 $\nu^z_{xy}$ 小一个数量级。 (b) $\mathcal{D}^y_{xy}(E) c_1[g(E/k_B T)]$ 随 $k_B T/J$ 和 $E/J$ 的变化。图中垂直虚线对应 $k_B T/J=0.58$,$k_B T/J=1.05$ 和 $k_B T/J=1.75$。图片来自\cite{lu2024Spin}。}
            \label{fig:coe_with_T}
        \end{figure}
    
        整体而言,随着温度升高,味能斯特系数的数值增加,因为方程~\eqref{eq:nu} 中的 $c_1$ 函数在较高温度下取值更大。然而,随温度升高,高能带的自旋Berry曲率贡献变得更加不可忽略,从而导致 $\nu^y_{xy}$ 呈现非单调行为。
        
        为了更详细地理解温度效应,我们在图~\ref{fig:colored dispersion and DOsB} 中通过
        \begin{equation}
            L(\symup{\tilde{\Omega}}^s_{xy})=\text{sgn}(\symup{\tilde{\Omega}}^s_{xy})\ln(1+|\symup{\tilde{\Omega}}^s_{xy}|)
        \end{equation}
        对能带进行着色,其中 $L(x)=\text{sgn}(x)\ln{(1+|x|)}$ 为对数函数。结果显示,自旋Berry曲率的非零值主要集中在反交叉点处,并且两个能带的自旋Berry曲率符号相反。

        \begin{figure}[b]
            \centering
            \includegraphics[width=\textwidth]{Fig4-DOsB.pdf}
            \caption{由对数化的自旋 Berry 曲率 $L(\tilde{\Omega}^s_n)$ 着色的色散关系,以及 DOS $\mathcal{D}(E)$ 和 DOsB $\mathcal{D}^s_{\alpha \beta}(E)$。(a) 为 $s=y$ 的情况,(b) 为 $s=z$ 的情况。图片来自\cite{lu2024Spin}。}
            \label{fig:colored dispersion and DOsB}
        \end{figure}
        
        在图~\ref{fig:colored dispersion and DOsB} 中,我们还展示了\emph{态密度}(Density of states, DOS) $\mathcal{D}(E)$ 以及\emph{自旋Berry曲率密度}(Density of spin Berry curvature, DOsB) $\mathcal{D}^s_{\alpha\beta}(E)$,其定义如下:
        \begin{align}
            \mathcal{D}(E)&= \int_{BZ} \sum_n \frac{d{\bm{k}}}{(2\pi)^2} \delta(E - E_n({\bm{k}})),\\
            \mathcal{D}^s_{\alpha\beta}(E)&= \int_{BZ} \sum_n \frac{d{\bm{k}}}{(2\pi)^2} \delta(E - E_n({\bm{k}})) \symup{\tilde{\Omega}}^s_{\alpha\beta,n{\bm{k}}}.
        \end{align}
        由于低能带在玻色子统计下对 味能斯特效应 贡献更为显著,并且最低的四条能带在能量上与更高能带分离,因此我们主要关注这四条能带。可以清晰地看到在4.8$J$以下DOsB呈现交替符号的显著变化,从而导致 $\nu^y_{xy}$ 的非单调行为。
        
        在数值计算 DOS 和 DOsB 时,由于程序上无法实现 $\delta(x)$ 函数,故我们选择使用另一个函数来近似 $\delta(x)$ 函数。在本文中,我们选取
        \begin{equation}
            \delta(x-x_0) \sim \frac{1}{2\sigma } e^{-\frac{\pi ( x-x_{0})^{2}}{\sigma ^{2}}}
        \end{equation}
        这是一个变形的高斯函数。当 $\sigma$ 较小的时候,上式有着非常好的近似效果。但是需要注意,$\sigma$ 不是越小越好,需要选择适当的值。如果 $\sigma$ 过小,会导致图像不光滑,产生很多锯齿。如果 $\sigma$ 过大,则无法很好地满足 $\delta$ 函数的筛选性质,使得计算不够准确。经过估算,$\delta$ 函数的宽度在 $2dk$ 时是最佳的。在本文中,我们取 $\sigma = 1.5 \times 10^{-2}$.

        从图~\ref{fig:coe_with_T}(a) 可以看出,$\nu^y_{xy}$ 在达到某一温度后开始显著增加,并在 $k_B T/J=0.58$ 处达到最大增长速率。随后,随着温度升高,该系数在 $k_B T/J=1.05$ 处达到峰值,然后下降,在 $k_B T/J=1.75$ 处达到局部最小值,之后又逐渐回升。为了解释 $\nu^y_{xy}$ 的行为,我们在图~\ref{fig:coe_with_T}(b) 中绘制了 $\mathcal{D}^y_{xy}(E) c_1[g(E/k_B T)]$ 随 $k_B T/J$ 和 $E/J$ 的变化情况。对于 $\mathcal{D}^y_{xy}(E)$,在 $3.34 <E/J <3.79$ 范围内主要为正(红色),在 $3.79< E/J <4.41$ 范围内主要为负(蓝色),如图~\ref{fig:colored dispersion and DOsB} 所示。因此,我们利用这些关键数据点在图~\ref{fig:coe_with_T}(b) 中用虚线标记了几个区域。当 $k_B T/J<0.58$ 时,我们在任意能量水平都未观察到蓝色区域。当温度处于 $0.58< k_B T/J <1.05$ 时,负区域IV减缓了 $\nu^y_{xy}$ 的增长。当温度进入 $1.05< k_B T/J <1.75$ 区间时,负区域V(尤其是围绕 $E/J=4$ 的深蓝区域)逆转了 $\nu^y_{xy}$ 的增长趋势。随着温度进一步升高,区域IX和III逐渐变红,导致 $\nu^y_{xy}$ 恢复增长趋势。

        
    \section{味能斯特系数与各项异性相互作用的变化关系}
        除了 $k_B T$ 之外,另一个特征能量尺度是各向异性 $\eta$,它决定了从顺磁基态到晶体场激发态的能隙。如图~\ref{fig:D0.06_B0.6_eta} 所示,较小的 $\eta$ 使得 CEF 能级更容易被热激发,从而导致更强的热响应。以焦绿石材料的典型晶格常数 $d\sim 10$ \AA~\cite{wen2021epitaxial} 作为薄膜厚度的估计,并考虑温度梯度 $\nabla_z T\sim 10$ K/mm~\cite{lin2022Evidence} 及图~\ref{fig:D0.06_B0.6_eta} 所采用的相同参数,可得到实验可观测的 $z(y)$ 极化自旋流,其大小可达 $10^{-11}$ ($10^{-12}$) J/m$^2$。

        \begin{figure}[th]
            \centering
            \includegraphics[width=0.9\textwidth]{Fig5-etaFNC.pdf}
            \caption{在 $k_B T/J=2$ 时,味能斯特系数(单位为 $k_B$)随 $\eta$ 的变化情况。其中 $D/J=0.06$,$B/J=0.6$。图片来自\cite{lu2024Spin}。}
            \label{fig:D0.06_B0.6_eta}
        \end{figure}
        
        显然,除了能量尺度的影响之外,能带的自旋 Berry 曲率本身也会显著影响 味能斯特效应。如前所述,在共线情况下,自旋 Berry 曲率与 Berry 曲率直接相关,因此即使在更一般的非共线情况下,味能斯特效应 也应当反映能带的几何结构。例如,如果重现由动量分辨的自旋 $z$ 分量 $S^z_{n{\bm{k}}} \equiv \langle \psi_{n{\bm{k}}} | S^z | \psi_{n{\bm{k}}} \rangle$ 进行着色的能带色散(如图~\ref{fig:Sz-Dispersion} 所示),则可以观察到,较低的三个能带主要为自旋向上态。因此,自旋 Berry 曲率可近似由能带 Berry 曲率给出:
        \begin{equation}
            \tilde{\Omega}^z_{xy,n{\bm{k}}} \sim \left( \frac{1}{V} \sum_{\bm{k}} S^z_{n{\bm{k}}} \right) \Omega_{xy,n{\bm{k}}},
        \end{equation}
        从而 味能斯特效应 间接地反映了能带的几何与拓扑特性。

        \begin{figure}[bht]
            \centering
            \includegraphics[width=0.9\textwidth]{Fig6-Sz_Dispersion.pdf}
            \caption{由动量分辨的自旋 $z$ 分量 $\bra{\psi_{n{\bm{k}}}} S^z \ket{\psi_{n{\bm{k}}}}$ 进行着色的能带色散关系。图片来自\cite{lu2024Spin}。}
            \label{fig:Sz-Dispersion}
        \end{figure}
        
    \section{味能斯特系数与DM相互作用的变化关系}
        由于Dzyaloshinskii-Moriya(DM)相互作用可以显著改变能带拓扑,我们在图~\ref{fig:dmi_with_t} 中研究了不同DM相互作用强度 $D$ 下 $\nu^{z}_{xy}$ 随温度 $T$ 的变化关系。

        \begin{figure}[th]
            \centering
            \includegraphics[width=0.83\textwidth]{figures/Fig7-DFNC.pdf}
            \caption{不同参数下的能斯特系数。我们设定 $\eta/J=6$,$B/J=0.6$,并考察不同 $D/J$ 的情况。同时,我们标注了最低四条能带的Chern数。图片来自\cite{lu2024Spin}。}
            \label{fig:dmi_with_t}
        \end{figure}
        
        对于每条曲线,从左到右的四个数值表示自底向上最低四条能带的Chern数,并且这些数值与曲线的行为相匹配。例如,当 ${D/J=-0.2}$ 或 $-0.1$ 时,最低能带具有正的Chern数 $+3$,从而导致正的能斯特系数。然而,与 $D/J=-0.2$ 时产生的较大正响应相比,在 $D/J=-0.1$ 的情况下,由于第二条能带的Chern数为 $-3$,较大的负值抑制了 $\nu^z_{xy}$ 随温度升高的增长。
        
        相反,当 $D/J=0$,$0.1$ 和 $0.2$ 时,最低能带的Chern数为 $-1$,从而导致负的能斯特系数。值得注意的是,由于伊辛轴的非共线性,即使在 $D=0$ 时,能带的Chern数以及能斯特系数仍然不为零。由于DM相互作用的微小偏差能够显著改变能带拓扑及热响应,因此在应变工程的帮助下,可以较容易地控制热自旋电流,特别是在层状系统或异质结构中~\cite{kim2020StrainEngineeringMagnetic,zhang2021strain,xu2022strain}。在这些情况下,沿层状方向施加的应变不仅可以调控体相或界面DM矢量,还能调整伊辛轴的方向。因此,我们的理论为操控热自旋电流提供了一条潜在路径,并可能在压电自旋电子学(piezo-spintronics)领域找到应用~\cite{nunez2014theory,ulloa2017piezospintronic,liu2019antiferromagnetic}。
        
    \section{味能斯特系数与外磁场的变化关系}
        由于磁场不仅可以改变能带色散关系,还会影响自旋Berry曲率的分布,因此味能斯特效应也可以通过磁场进行调控。然而,与热霍尔效应不同的是,反转磁场并不会翻转热自旋电流的响应,这是因为自旋Berry曲率在时间反演对称性下保持不变。

        在图~\ref{fig:Dz6_D0.06_B_T} 中,我们展示了外加磁场 $B$(以 $J$ 为单位)下 $
        nu^y_{xy}$ 和 $
        nu^z_{xy}$ 在不同温度下的变化关系。虚线部分是在 $B=0$ 附近的外推数据,以节省在极小能隙下的计算时间。由于在小磁场趋于零的过程中,没有其他能隙闭合或拓扑相变发生,因此这些外推数据应当能够很好地反映该系数变化的趋势。我们的计算结果表明,响应系数 $
        nu^y_{xy}$ 和 $
        nu^z_{xy}$ 确实是 $B$ 的偶函数,并且当 $B\to0$ 时,它们的取值不一定趋于零。
        \begin{figure}[htp]
            \centering
            \includegraphics[width=0.6\textwidth]{figures/Fig8-BFNC.pdf}
            \caption{不同温度 $k_B T/J$ 下的 flavor Nernst 系数随磁场 $B$ 的变化。(a) 对应于 $\nu^y_{xy}$,(b) 对应于 $\nu^z_{xy}$。参数取值为 $\eta/J=6$,$D/J=0.06$。虚线表示计算数据的外推结果。图片来自\cite{lu2024Spin}。}
            \label{fig:Dz6_D0.06_B_T}
        \end{figure}

    \section{本章小结}

    在本章中,我们系统研究了二维 \textit{pyrochlore} 薄层上的自旋能斯特效应,并深入探讨了味能斯特系数随温度、各项异性相互作用、DM 相互作用以及外磁场的变化规律及其物理机制。

    首先,我们在非共线伊辛轴配置下,分析了味能斯特系数 $\nu^y_{xy}$ 和 $\nu^z_{xy}$ 随温度的变化特征。研究表明,$\nu^y_{xy}$ 虽非零,但其数值比 $\nu^z_{xy}$ 小一个数量级,这源于自旋非守恒效应的影响。通过计算自旋 Berry 曲率在能带中的分布,并结合态密度(DOS)和自旋 Berry 曲率密度(DOsB)的分析,我们揭示了 $\nu^y_{xy}$ 非单调行为的来源,即低能带和高能带的自旋 Berry 曲率贡献随温度变化的竞争效应。

    其次,我们考察了味能斯特系数与各项异性相互作用 $\eta$ 的依赖关系。结果显示,较小的 $\eta$ 减小了晶体场激发能隙,使能级更容易被热激发,从而显著增强热响应。同时,自旋 Berry 曲率与能带几何结构的关联表明,味能斯特效应间接反映了系统的拓扑特性。

    接着,我们研究了 Dzyaloshinskii-Moriya(DM)相互作用对味能斯特系数的影响。DM 相互作用通过改变能带拓扑结构显著调控了 $\nu^z_{xy}$ 的行为。我们发现,不同 DM 强度下 $\nu^z_{xy}$ 随温度的变化与最低四条能带的 Chern 数密切相关,为通过应变工程操控热自旋电流提供了理论依据。

    最后,我们分析了外磁场对味能斯特系数的影响。计算结果表明,$\nu^y_{xy}$ 和 $\nu^z_{xy}$ 是磁场 $B$ 的偶函数,且在 $B \to 0$ 时不一定趋于零。这一特性源于自旋 Berry 曲率在时间反演对称性下的不变性,区别于热霍尔效应的行为。

    综上所述,本章通过数值计算和理论分析,揭示了二维 \textit{pyrochlore} 薄层中自旋能斯特效应的多场调控机制,阐明了温度、各项异性、DM 相互作用和外磁场对味能斯特系数的综合影响。研究成果不仅加深了我们对自旋能斯特效应物理本质的理解,还为通过外部参数调控热自旋电流提供了理论支持,具有潜在的压电自旋电子学应用价值。